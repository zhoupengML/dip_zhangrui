\documentclass[journal,comsoc]{IEEEtran}
%\usepackage[UTF8, heading = false, scheme = plain]{ctex}
\usepackage[T1]{fontenc}% optional T1 font encoding
\usepackage{url}
\usepackage{graphicx}
\usepackage{caption}
\usepackage{subcaption}
\usepackage{amsmath}
\interdisplaylinepenalty=2500
\usepackage[cmintegrals]{newtxmath}
\hyphenation{op-tical net-works semi-conduc-tor}



\makeatletter
    \setlength\@fptop{0\p@}
\makeatother

\begin{document}
\title{Image Denosing with Wavelet-based Algorithms}
\author{Zhou~Peng,~\IEEEmembership{016034910052,~SEIEE~5-321,}}
\maketitle

\begin{abstract}
  In the real world, signals do not exist without noise. The Gaussian noise
  is the most typical noise which exists in our life. In this report, I will
  exploit wavelet denoising applied in image processing. I added different
  amount of the Gaussian noise by changing its standard deviation, and tested
  the performance of denosing using two methods:hard threshold and soft threshold.
  To find the best threshold for both hard method and soft method, I have tested
  many defferent thresholds and computed the signal to noise ratio(SNR) for recovered
  images. Our results shows that the best value of threshold for hard threshold is about
  at $3\sigma$ and for soft threshold is about at $1.5\sigma$ where the $\sigma$ is
  standard deviation of the added Gaussian noise. I also find the denoised image
  of soft threshold is more smooth than the denoised images of hard threshold.
\end{abstract}

% Note that keywords are not normally used for peerreview papers.
\begin{IEEEkeywords}
Wavelet denoising,Gaussian noise,hard threshold,soft threshold
\end{IEEEkeywords}

\section{Introduction}

\IEEEPARstart{T}{he} wavelet transform is similar to the Fourier transform with a completely different merit function.
The main difference is that Fourier transform decomposes the signal into sines and cosines, in contrary
the wavelet transform uses functions that are localized in both the real and Fourier space.
Generally, the wavelet transform can be expressed by the following equation:
\begin{equation}
  F(a,b)=\int_{-\infty}^{\infty}f(x)\psi_{(a,b)}^*(x)dx
\end{equation}
where the $*$ is the complex conjugate symbol and function $\psi$ is some function. 

As it is seen, the Wavelet transform is in fact an infinite set of various transforms, depending on the
merit function used for its computation. This is the main reason why we can hear the term wavelet transform
in very different situations and applications. 


% \section{Related Work}
% \label{sec:related work}

\section{Wavelet-based Denoising Algorithms}
\label{sec:wavelet-based-algorithms}

\subsection{Wavelet Denoising}
\label{sec:wavelet-denoising}

As we all know that signals do not exist without noise in the real world.
Under ideal conditions, this noise may decrease to such negligible
levels, while the signal increases to such significant levels, that for all practical purposes denoising
is not necessary. Unfortunately, the noise corrupting the signal, more often than not, must be
removed in order to recover the signal and proceed with further data analysis.

The principle of wavelet denoising\cite{wavelet} is that we can convert the original signal from time-space domain to
the time-scale domain via the wavelet transform. And in the time-scale domain, the noise can be eliminated
easily. There are so many different methods for wavelet denoising. In this report, I will mainly focus on
the hard threshold method and the soft threshold method.

\subsection{Hard Thresholding in Wavelet Bases}
\label{subsec:hard-thresholding-in-wavelet-bases}

A simple but efficient non-linear denoising estimator is obtained by thresholding the coefficients of $f$
in a well chosen orthogonal basis.
In the following, we will focuss on a wavelet basis, which is efficient to denoise piecewise regular images.
The hard thresholding operator with threshold $T\ge{0}$ applied to some image $f$ is defined as
\begin{eqnarray}
  S_T^0(f)&=&\sum_{\mid{<f,\psi_m>\mid}>T}<f,\psi_m>\psi_m\\
  &=&\sum_{m}s_T^0{(<f,\psi_m>)\psi_m}\nonumber
\end{eqnarray}
where the hard thresholding operator is
\begin{eqnarray}
  s_T^{0}(\alpha)=
  \begin{cases}
    \alpha\ \ \ if\ \mid{\alpha}\mid>T\\
    0\ \ \ otherwise
  \end{cases}
\end{eqnarray}
Fig. \ref{fig:hard-threshold-function} illustrates $s_T^0{(\alpha)}$ with $T=1$.
\begin{figure}[!htb]
  \centering
  \includegraphics[width=0.4\textwidth]{./code/results/hard_threshold.eps}
  \caption{Hard threshold function}
  \label{fig:hard-threshold-function}    
\end{figure}

The denoising estimator is then defined as
\begin{equation}
  \tilde{f}=S_T^0(f)
\end{equation}

\subsection{Soft Thresholding in Wavelet Bases}
\label{subsec:soft-thresholding-inwavelet-bases}

The estimated image $\tilde{f}$ using hard thresholding suffers from many artifacts.
It is possible to improve the result by using soft thresholding, defined as
\begin{equation}
  \tilde{f}=S_T^1(f)=\sum_ms_T^1(<f,\psi_m>)\psi_m
\end{equation}
where
\begin{equation}
  s_T^1(\alpha)=\max{(0,1-\frac{T}{\mid{\alpha}\mid}})\alpha
\end{equation}
Fig. \ref{fig:soft-threshold-function} illustrates $s_T^1{(\alpha)}$ with $T=1$.
\begin{figure}[!htb]
  \centering
  \includegraphics[width=0.4\textwidth]{./code/results/soft_threshold.eps}
  \caption{Soft threshold function}
  \label{fig:soft-threshold-function}    
\end{figure}


\section{Experiment}
\label{sec:experiment}

\subsection{Gaussian Noise}
\label{subsec:gaussian-noise}

For real life applications, we cann't have access to the underlying image $f_0$. We assume that $f_0$
is known, and $f=f_0+w$ is generated using a single realization of the noise $w$. We define the estimated image
as $\tilde{f}$ which is the denoised image. Fig. \ref{fig:add-gaussian-noise} shows images which were added
Gaussian noise with standard deviation $0.1$. 
%\usepackage{subcaption}
\begin{figure}[!htb]
  \centering
  \begin{subfigure}[t]{.25\textwidth}
    \centering
    \includegraphics[width=0.68\textwidth]{./code/results/lena512.eps}
    \caption{Lena}
    \label{subfig:lena}
  \end{subfigure}%
  \begin{subfigure}[t]{.25\textwidth}
    \centering
    \includegraphics[width=0.68\textwidth]{./code/results/lena512_gaussian_0_1.eps}
    \caption{Gaussian noisy image of (a),\ $\sigma=0.1$}
    \label{subfig:gaussian-noisy-image-of-a}
  \end{subfigure}
  \begin{subfigure}[t]{0.25\textwidth}
    \centering
    \includegraphics[width=0.68\textwidth]{./code/results/flowr.eps}
    \caption{Flower}
    \label{subfig:flower}
  \end{subfigure}%
  \begin{subfigure}[t]{0.25\textwidth}
    \centering
    \includegraphics[width=0.68\textwidth]{./code/results/flowr_gaussian_0_1.eps}
    \caption{Gaussian noisy image of (c),\ $\sigma=0.1$}
    \label{subfig:gaussian-noisy-image-of-c}
  \end{subfigure}
  \caption{Add Gaussian noise}
  \label{fig:add-gaussian-noise}
\end{figure}

\subsection{Denoising with Hard Threshold}
\label{subsec:denoising-with-hard-thresholding}

I performed several denoising experiment by adding different amount of the Gaussian
noise. I converted the gray value of images in the range of $[0,1]$. Setting the
standard deviation of the Gaussian noise in $[0.1,0.2,0.3]$, I tested the performance of
wavelet denoising based on hard threshold. 
I tested several hard threshold in $[0.1\sigma,0.15\sigma,0.3\sigma,0.45\sigma]$,
where $\sigma$ is the standard deviation of the added Gaussian noise.
Fig. \ref{fig:sigma-10-hard-threshold} - Fig. \ref{fig:flower-sigma-30-hard-threshold} shows our results.
We can find that the wavelet denoising algorithm based on hard threshold is effective to
denoise Gaussian noise. And the best hard threshold is approximately at $0.3\sigma$.



\subsection{Denoising with Soft Threshold}
\label{subsec:denoising-with-soft-thresholding}

Analogous to hard threshold denoising algorithm, I keep other conditions invariant, and
change the hard threshold to soft threshold. Fig. \ref{fig:sigma-10-soft-threshold} -
Fig. \ref{fig:flower-sigma-30-soft-threshold} shows the results. From the results, we can
easily find that the best soft threshold is approximately at $0.15\sigma$. When the
soft threshold is too small, the Gaussian noise can't be eliminated effectively. But
if the soft threshold is too big, the recovered images will become more fuzzy.
Compared with the best denoising images using hard threshold, the soft threshold seems
to make the denoising images more smooth.



\subsection{Finding the Best Threshold}
\label{subsec:finding-the-best-threshold}

%\usepackage{subcaption}
\begin{figure}[!hbt]
  \centering
  \begin{subfigure}{.36\textwidth}
    \centering
    \includegraphics[width=\textwidth]{./code/results/lena512_snr.eps}
    \caption{SNR with different thresholds of Lena, $\sigma=0.1$}
    \label{subfig:snr-threshold-lena}
  \end{subfigure}
  \begin{subfigure}{.36\textwidth}
    \centering
    \includegraphics[width=\textwidth]{./code/results/flowr_snr.eps}
    \caption{SNR with different thresholds of Flower, $\sigma=0.1$}
    \label{subfig:snr-threshold-flower}
  \end{subfigure}
  \caption{SNR vs Threshold}
  \label{fig:snr-lena-flower}
\end{figure}

We define the empirical signal to noise ratio as
\begin{eqnarray}
  SNR=20\log_{10}(\frac{\|f_0\|}{\|f_0-\tilde{f_0}\|})
\end{eqnarray}
where $f_0$ is the original image and $\tilde{f_0}$ is the recovered image.

To determine the best threshold $T$ for both hard and soft thresholding, I tested several
$T$ values in $[0.1\sigma,6\sigma]$. I tested two images(Lena and Flower) which were added gaussian noise
whose standard deviation is $0.1$. Fig. \ref{fig:snr-lena-flower} shows the experiment results.
From the two images, we can estimate that the best threshold for hard threshold is about $3\sigma$
and for soft threshold is about $1.5\sigma$.



% hard threshold
%\usepackage{subcaption}
\begin{figure}[!hbt]
  \centering
  \begin{subfigure}{.25\textwidth}
    \centering
    \includegraphics[width=0.68\textwidth]{./code/results/lena512.eps}
    \caption{Original image of Lena}
    \label{subfig:original-image-of-lena}
  \end{subfigure}%
  \begin{subfigure}{.25\textwidth}
    \centering
    \includegraphics[width=0.68\textwidth]{./code/results/lena512_noise_sigma_10.eps}
    \caption{Noisy image of (a)}
    \label{subfig:sigma-10-noisy-image-of-a}
  \end{subfigure}

  \begin{subfigure}{0.25\textwidth}
    \centering{}
    \includegraphics[width=.68\textwidth]{./code/results/lena512_sigma_10_hard_1.eps}
    \caption{Hard threshold: $0.1\sigma$}
  \end{subfigure}%
  \begin{subfigure}{.25\textwidth}
    \centering{}
    \includegraphics[width=.68\textwidth]{./code/results/lena512_sigma_10_hard_15.eps}
    \caption{Hard threshold: $0.15\sigma$}
  \end{subfigure}

  \begin{subfigure}{0.25\textwidth}
    \centering{}
    \includegraphics[width=.68\textwidth]{./code/results/lena512_sigma_10_hard_30.eps}
    \caption{Hard threshold: $0.3\sigma$}
  \end{subfigure}%
  \begin{subfigure}{.25\textwidth}
    \centering{}
    \includegraphics[width=.68\textwidth]{./code/results/lena512_sigma_10_hard_45.eps}
    \caption{Hard threshold: $0.45\sigma$}
  \end{subfigure}
  
  \caption{Denoising using hard threshold, $\sigma=0.1$}
  \label{fig:sigma-10-hard-threshold}
\end{figure}

\begin{figure}[!hbt]
  \centering
  \begin{subfigure}{.25\textwidth}
    \centering
    \includegraphics[width=0.68\textwidth]{./code/results/lena512.eps}
    \caption{Original image of Lena}
    \label{subfig:original-image-of-lena}
  \end{subfigure}%
  \begin{subfigure}{.25\textwidth}
    \centering
    \includegraphics[width=0.68\textwidth]{./code/results/lena512_noise_sigma_20.eps}
    \caption{Noisy image of (a)}
    \label{subfig:sigma-10-noisy-image-of-a}
  \end{subfigure}

  \begin{subfigure}{0.25\textwidth}
    \centering{}
    \includegraphics[width=.68\textwidth]{./code/results/lena512_sigma_20_hard_1.eps}
    \caption{Hard threshold: $0.1\sigma$}
  \end{subfigure}%
  \begin{subfigure}{.25\textwidth}
    \centering{}
    \includegraphics[width=.68\textwidth]{./code/results/lena512_sigma_20_hard_15.eps}
    \caption{Hard threshold: $0.15\sigma$}
  \end{subfigure}

  \begin{subfigure}{0.25\textwidth}
    \centering{}
    \includegraphics[width=.68\textwidth]{./code/results/lena512_sigma_20_hard_30.eps}
    \caption{Hard threshold: $0.3\sigma$}
  \end{subfigure}%
  \begin{subfigure}{.25\textwidth}
    \centering{}
    \includegraphics[width=.68\textwidth]{./code/results/lena512_sigma_20_hard_45.eps}
    \caption{Hard threshold: $0.45\sigma$}
  \end{subfigure}
  
  \caption{Denoising using hard threshold, $\sigma=0.2$}
  \label{fig:sigma-20-hard-threshold}
\end{figure}

\begin{figure}[!hbt]
  \centering
  \begin{subfigure}{.25\textwidth}
    \centering
    \includegraphics[width=0.68\textwidth]{./code/results/lena512.eps}
    \caption{Original image of Lena}
    \label{subfig:original-image-of-lena}
  \end{subfigure}%
  \begin{subfigure}{.25\textwidth}
    \centering
    \includegraphics[width=0.68\textwidth]{./code/results/lena512_noise_sigma_30.eps}
    \caption{Noisy image of (a)}
    \label{subfig:sigma-10-noisy-image-of-a}
  \end{subfigure}

  \begin{subfigure}{0.25\textwidth}
    \centering{}
    \includegraphics[width=.68\textwidth]{./code/results/lena512_sigma_30_hard_1.eps}
    \caption{Hard threshold: $0.1\sigma$}
  \end{subfigure}%
  \begin{subfigure}{.25\textwidth}
    \centering{}
    \includegraphics[width=.68\textwidth]{./code/results/lena512_sigma_30_hard_15.eps}
    \caption{Hard threshold: $0.15\sigma$}
  \end{subfigure}

  \begin{subfigure}{0.25\textwidth}
    \centering{}
    \includegraphics[width=.68\textwidth]{./code/results/lena512_sigma_30_hard_30.eps}
    \caption{Hard threshold: $0.3\sigma$}
  \end{subfigure}%
  \begin{subfigure}{.25\textwidth}
    \centering{}
    \includegraphics[width=.68\textwidth]{./code/results/lena512_sigma_30_hard_45.eps}
    \caption{Hard threshold: $0.45\sigma$}
  \end{subfigure}
  
  \caption{Denoising using hard threshold, $\sigma=0.3$}
  \label{fig:sigma-30-hard-threshold}
\end{figure}


% second image
\begin{figure}[!hbt]
  \centering
  \begin{subfigure}{.25\textwidth}
    \centering
    \includegraphics[width=0.68\textwidth]{./code/results/flowr.eps}
    \caption{Original image of Flower}
    \label{subfig:original-image-of-flower}
  \end{subfigure}%
  \begin{subfigure}{.25\textwidth}
    \centering
    \includegraphics[width=0.68\textwidth]{./code/results/flowr_noise_sigma_10.eps}
    \caption{Noisy image of (a)}
    \label{subfig:sigma-10-noisy-image-of-a}
  \end{subfigure}

  \begin{subfigure}{0.25\textwidth}
    \centering{}
    \includegraphics[width=.68\textwidth]{./code/results/flowr_sigma_10_hard_1.eps}
    \caption{Hard threshold: $0.1\sigma$}
  \end{subfigure}%
  \begin{subfigure}{.25\textwidth}
    \centering{}
    \includegraphics[width=.68\textwidth]{./code/results/flowr_sigma_10_hard_15.eps}
    \caption{Hard threshold: $0.15\sigma$}
  \end{subfigure}

  \begin{subfigure}{0.25\textwidth}
    \centering{}
    \includegraphics[width=.68\textwidth]{./code/results/flowr_sigma_10_hard_30.eps}
    \caption{Hard threshold: $0.3\sigma$}
  \end{subfigure}%
  \begin{subfigure}{.25\textwidth}
    \centering{}
    \includegraphics[width=.68\textwidth]{./code/results/flowr_sigma_10_hard_45.eps}
    \caption{Hard threshold: $0.45\sigma$}
  \end{subfigure}
  
  \caption{Denoising using hard threshold, $\sigma=0.1$}
  \label{fig:flower-sigma-10-hard-threshold}
\end{figure}

\begin{figure}[!hbt]
  \centering
  \begin{subfigure}{.25\textwidth}
    \centering
    \includegraphics[width=0.68\textwidth]{./code/results/flowr.eps}
    \caption{Original image of Flower}
    \label{subfig:original-image-of-flower}
  \end{subfigure}%
  \begin{subfigure}{.25\textwidth}
    \centering
    \includegraphics[width=0.68\textwidth]{./code/results/flowr_noise_sigma_20.eps}
    \caption{Noisy image of (a)}
    \label{subfig:flower-sigma-10-noisy-image-of-a}
  \end{subfigure}

  \begin{subfigure}{0.25\textwidth}
    \centering{}
    \includegraphics[width=.68\textwidth]{./code/results/flowr_sigma_20_hard_1.eps}
    \caption{Hard threshold: $0.1\sigma$}
  \end{subfigure}%
  \begin{subfigure}{.25\textwidth}
    \centering{}
    \includegraphics[width=.68\textwidth]{./code/results/flowr_sigma_20_hard_15.eps}
    \caption{Hard threshold: $0.15\sigma$}
  \end{subfigure}

  \begin{subfigure}{0.25\textwidth}
    \centering{}
    \includegraphics[width=.68\textwidth]{./code/results/flowr_sigma_20_hard_30.eps}
    \caption{Hard threshold: $0.3\sigma$}
  \end{subfigure}%
  \begin{subfigure}{.25\textwidth}
    \centering{}
    \includegraphics[width=.68\textwidth]{./code/results/flowr_sigma_20_hard_45.eps}
    \caption{Hard threshold: $0.45\sigma$}
  \end{subfigure}
  
  \caption{Denoising using hard threshold, $\sigma=0.2$}
  \label{fig:flower-sigma-20-hard-threshold}
\end{figure}

\begin{figure}[!hbt]
  \centering
  \begin{subfigure}{.25\textwidth}
    \centering
    \includegraphics[width=0.68\textwidth]{./code/results/flowr.eps}
    \caption{Original image of Flower}
    \label{subfig:original-image-of-flower}
  \end{subfigure}%
  \begin{subfigure}{.25\textwidth}
    \centering
    \includegraphics[width=0.68\textwidth]{./code/results/flowr_noise_sigma_30.eps}
    \caption{Noisy image of (a)}
    \label{subfig:flower-sigma-10-noisy-image-of-a}
  \end{subfigure}

  \begin{subfigure}{0.25\textwidth}
    \centering{}
    \includegraphics[width=.68\textwidth]{./code/results/flowr_sigma_30_hard_1.eps}
    \caption{Hard threshold: $0.1\sigma$}
  \end{subfigure}%
  \begin{subfigure}{.25\textwidth}
    \centering{}
    \includegraphics[width=.68\textwidth]{./code/results/flowr_sigma_30_hard_15.eps}
    \caption{Hard threshold: $0.15\sigma$}
  \end{subfigure}

  \begin{subfigure}{0.25\textwidth}
    \centering{}
    \includegraphics[width=.68\textwidth]{./code/results/flowr_sigma_30_hard_30.eps}
    \caption{Hard threshold: $0.3\sigma$}
  \end{subfigure}%
  \begin{subfigure}{.25\textwidth}
    \centering{}
    \includegraphics[width=.68\textwidth]{./code/results/flowr_sigma_30_hard_45.eps}
    \caption{Hard threshold: $0.45\sigma$}
  \end{subfigure}
  
  \caption{Denoising using hard threshold, $\sigma=0.3$}
  \label{fig:flower-sigma-30-hard-threshold}
\end{figure}


% Soft threshold
\begin{figure}[!hbt]
  \centering
  \begin{subfigure}{.25\textwidth}
    \centering
    \includegraphics[width=0.68\textwidth]{./code/results/lena512.eps}
    \caption{Original image of Lena}
    \label{subfig:original-image-of-lena}
  \end{subfigure}%
  \begin{subfigure}{.25\textwidth}
    \centering
    \includegraphics[width=0.68\textwidth]{./code/results/lena512_noise_sigma_10.eps}
    \caption{Noisy image of (a)}
    \label{subfig:sigma-10-noisy-image-of-a}
  \end{subfigure}

  \begin{subfigure}{0.25\textwidth}
    \centering{}
    \includegraphics[width=.68\textwidth]{./code/results/lena512_sigma_10_soft_1.eps}
    \caption{Soft threshold: $0.1\sigma$}
  \end{subfigure}%
  \begin{subfigure}{.25\textwidth}
    \centering{}
    \includegraphics[width=.68\textwidth]{./code/results/lena512_sigma_10_soft_15.eps}
    \caption{Soft threshold: $0.15\sigma$}
  \end{subfigure}

  \begin{subfigure}{0.25\textwidth}
    \centering{}
    \includegraphics[width=.68\textwidth]{./code/results/lena512_sigma_10_soft_30.eps}
    \caption{Soft threshold: $0.3\sigma$}
  \end{subfigure}%
  \begin{subfigure}{.25\textwidth}
    \centering{}
    \includegraphics[width=.68\textwidth]{./code/results/lena512_sigma_10_soft_45.eps}
    \caption{Soft threshold: $0.45\sigma$}
  \end{subfigure}
  
  \caption{Denoising using soft threshold, $\sigma=0.1$}
  \label{fig:sigma-10-soft-threshold}
\end{figure}

\begin{figure}[!hbt]
  \centering
  \begin{subfigure}{.25\textwidth}
    \centering
    \includegraphics[width=0.68\textwidth]{./code/results/lena512.eps}
    \caption{Original image of Lena}
    \label{subfig:original-image-of-lena}
  \end{subfigure}%
  \begin{subfigure}{.25\textwidth}
    \centering
    \includegraphics[width=0.68\textwidth]{./code/results/lena512_noise_sigma_20.eps}
    \caption{Noisy image of (a)}
    \label{subfig:sigma-10-noisy-image-of-a}
  \end{subfigure}

  \begin{subfigure}{0.25\textwidth}
    \centering{}
    \includegraphics[width=.68\textwidth]{./code/results/lena512_sigma_20_soft_1.eps}
    \caption{Soft threshold: $0.1\sigma$}
  \end{subfigure}%
  \begin{subfigure}{.25\textwidth}
    \centering{}
    \includegraphics[width=.68\textwidth]{./code/results/lena512_sigma_20_soft_15.eps}
    \caption{Soft threshold: $0.15\sigma$}
  \end{subfigure}

  \begin{subfigure}{0.25\textwidth}
    \centering{}
    \includegraphics[width=.68\textwidth]{./code/results/lena512_sigma_20_soft_30.eps}
    \caption{Soft threshold: $0.3\sigma$}
  \end{subfigure}%
  \begin{subfigure}{.25\textwidth}
    \centering{}
    \includegraphics[width=.68\textwidth]{./code/results/lena512_sigma_20_soft_45.eps}
    \caption{Soft threshold: $0.45\sigma$}
  \end{subfigure}
  
  \caption{Denoising using soft threshold, $\sigma=0.2$}
  \label{fig:sigma-20-soft-threshold}
\end{figure}

\begin{figure}[!hbt]
  \centering
  \begin{subfigure}{.25\textwidth}
    \centering
    \includegraphics[width=0.68\textwidth]{./code/results/lena512.eps}
    \caption{Original image of Lena}
    \label{subfig:original-image-of-lena}
  \end{subfigure}%
  \begin{subfigure}{.25\textwidth}
    \centering
    \includegraphics[width=0.68\textwidth]{./code/results/lena512_noise_sigma_30.eps}
    \caption{Noisy image of (a)}
    \label{subfig:sigma-10-noisy-image-of-a}
  \end{subfigure}

  \begin{subfigure}{0.25\textwidth}
    \centering{}
    \includegraphics[width=.68\textwidth]{./code/results/lena512_sigma_30_soft_1.eps}
    \caption{Soft threshold: $0.1\sigma$}
  \end{subfigure}%
  \begin{subfigure}{.25\textwidth}
    \centering{}
    \includegraphics[width=.68\textwidth]{./code/results/lena512_sigma_30_soft_15.eps}
    \caption{Soft threshold: $0.15\sigma$}
  \end{subfigure}

  \begin{subfigure}{0.25\textwidth}
    \centering{}
    \includegraphics[width=.68\textwidth]{./code/results/lena512_sigma_30_soft_30.eps}
    \caption{Soft threshold: $0.3\sigma$}
  \end{subfigure}%
  \begin{subfigure}{.25\textwidth}
    \centering{}
    \includegraphics[width=.68\textwidth]{./code/results/lena512_sigma_30_soft_45.eps}
    \caption{Soft threshold: $0.45\sigma$}
  \end{subfigure}
  
  \caption{Denoising using soft threshold, $\sigma=0.3$}
  \label{fig:sigma-30-soft-threshold}
\end{figure}


% second image
\begin{figure}[!hbt]
  \centering
  \begin{subfigure}{.25\textwidth}
    \centering
    \includegraphics[width=0.68\textwidth]{./code/results/flowr.eps}
    \caption{Original image of Flower}
    \label{subfig:original-image-of-flower}
  \end{subfigure}%
  \begin{subfigure}{.25\textwidth}
    \centering
    \includegraphics[width=0.68\textwidth]{./code/results/flowr_noise_sigma_10.eps}
    \caption{Noisy image of (a)}
    \label{subfig:sigma-10-noisy-image-of-a}
  \end{subfigure}

  \begin{subfigure}{0.25\textwidth}
    \centering{}
    \includegraphics[width=.68\textwidth]{./code/results/flowr_sigma_10_soft_1.eps}
    \caption{Soft threshold: $0.1\sigma$}
  \end{subfigure}%
  \begin{subfigure}{.25\textwidth}
    \centering{}
    \includegraphics[width=.68\textwidth]{./code/results/flowr_sigma_10_soft_15.eps}
    \caption{Soft threshold: $0.15\sigma$}
  \end{subfigure}

  \begin{subfigure}{0.25\textwidth}
    \centering{}
    \includegraphics[width=.68\textwidth]{./code/results/flowr_sigma_10_soft_30.eps}
    \caption{Soft threshold: $0.3\sigma$}
  \end{subfigure}%
  \begin{subfigure}{.25\textwidth}
    \centering{}
    \includegraphics[width=.68\textwidth]{./code/results/flowr_sigma_10_soft_45.eps}
    \caption{Soft threshold: $0.45\sigma$}
  \end{subfigure}
  
  \caption{Denoising using soft threshold, $\sigma=0.1$}
  \label{fig:flower-sigma-10-soft-threshold}
\end{figure}

\begin{figure}[!hbt]
  \centering
  \begin{subfigure}{.25\textwidth}
    \centering
    \includegraphics[width=0.68\textwidth]{./code/results/flowr.eps}
    \caption{Original image of Flower}
    \label{subfig:original-image-of-flower}
  \end{subfigure}%
  \begin{subfigure}{.25\textwidth}
    \centering
    \includegraphics[width=0.68\textwidth]{./code/results/flowr_noise_sigma_20.eps}
    \caption{Noisy image of (a)}
    \label{subfig:flower-sigma-10-noisy-image-of-a}
  \end{subfigure}

  \begin{subfigure}{0.25\textwidth}
    \centering{}
    \includegraphics[width=.68\textwidth]{./code/results/flowr_sigma_20_soft_1.eps}
    \caption{Soft threshold: $0.1\sigma$}
  \end{subfigure}%
  \begin{subfigure}{.25\textwidth}
    \centering{}
    \includegraphics[width=.68\textwidth]{./code/results/flowr_sigma_20_soft_15.eps}
    \caption{Soft threshold: $0.15\sigma$}
  \end{subfigure}

  \begin{subfigure}{0.25\textwidth}
    \centering{}
    \includegraphics[width=.68\textwidth]{./code/results/flowr_sigma_20_soft_30.eps}
    \caption{Soft threshold: $0.3\sigma$}
  \end{subfigure}%
  \begin{subfigure}{.25\textwidth}
    \centering{}
    \includegraphics[width=.68\textwidth]{./code/results/flowr_sigma_20_soft_45.eps}
    \caption{Soft threshold: $0.45\sigma$}
  \end{subfigure}
  
  \caption{Denoising using soft threshold, $\sigma=0.2$}
  \label{fig:flower-sigma-20-soft-threshold}
\end{figure}

\begin{figure}[!hbt]
  \centering
  \begin{subfigure}{.25\textwidth}
    \centering
    \includegraphics[width=0.68\textwidth]{./code/results/flowr.eps}
    \caption{Original image of Flower}
    \label{subfig:original-image-of-flower}
  \end{subfigure}%
  \begin{subfigure}{.25\textwidth}
    \centering
    \includegraphics[width=0.68\textwidth]{./code/results/flowr_noise_sigma_30.eps}
    \caption{Noisy image of (a)}
    \label{subfig:flower-sigma-10-noisy-image-of-a}
  \end{subfigure}

  \begin{subfigure}{0.25\textwidth}
    \centering{}
    \includegraphics[width=.68\textwidth]{./code/results/flowr_sigma_30_soft_1.eps}
    \caption{Soft threshold: $0.1\sigma$}
  \end{subfigure}%
  \begin{subfigure}{.25\textwidth}
    \centering{}
    \includegraphics[width=.68\textwidth]{./code/results/flowr_sigma_30_soft_15.eps}
    \caption{Soft threshold: $0.15\sigma$}
  \end{subfigure}

  \begin{subfigure}{0.25\textwidth}
    \centering{}
    \includegraphics[width=.68\textwidth]{./code/results/flowr_sigma_30_soft_30.eps}
    \caption{Soft threshold: $0.3\sigma$}
  \end{subfigure}%
  \begin{subfigure}{.25\textwidth}
    \centering{}
    \includegraphics[width=.68\textwidth]{./code/results/flowr_sigma_30_soft_45.eps}
    \caption{Soft threshold: $0.45\sigma$}
  \end{subfigure}
  
  \caption{Denoising using soft threshold, $\sigma=0.3$}
  \label{fig:flower-sigma-30-soft-threshold}
\end{figure}


\section{Conclusion}

This report shows how to use wavelets to denoise images. Because wavelets localize
features in your data to different scales, you can preserve important image features
while removing noise. The basic idea behind wavelet denoising, or wavelet thresholding,
is that the wavelet transform leads to a sparse representation for many real-world images.
What this means is that the wavelet transform concentrates image features in a few
large-magnitude wavelet coefficients. Wavelet coefficients which are small in value are
typically noise and you can remove them without affecting the image quality. After you
threshold the coefficients, you can reconstruct the data using the inverse wavelet transform.

% use section* for acknowledgment
\section*{Acknowledgment}
I would like to thank Professor Zhang Rui for her instruction.

\begin{thebibliography}{1}
\bibitem{wavelet}
  Chang, S. Grace, Bin Yu, and Martin Vetterli. "Image denoising via lossy compression and wavelet thresholding."
  Image Processing, 1997. Proceedings., International Conference on. Vol. 1. IEEE, 1997.

\end{thebibliography}


% \begin{IEEEbiographynophoto}{Zhou Peng}
% received his bachelor degree from UESTC.He is now studying for a PhD in SJTU.
% \end{IEEEbiographynophoto}

\vfill

\end{document}


