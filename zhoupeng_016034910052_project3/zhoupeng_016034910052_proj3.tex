\documentclass[journal,comsoc]{IEEEtran}
%\usepackage[UTF8, heading = false, scheme = plain]{ctex}
\usepackage[T1]{fontenc}% optional T1 font encoding
\usepackage{url}
\usepackage{graphicx}
\usepackage{caption}
\usepackage{subcaption}
\usepackage{amsmath}
\interdisplaylinepenalty=2500
\usepackage[cmintegrals]{newtxmath}
\hyphenation{op-tical net-works semi-conduc-tor}

\makeatletter
    \setlength\@fptop{0\p@}
\makeatother

\begin{document}
\title{Image Denosing with Wavelet-based Algorithms}
\author{Zhou~Peng,~\IEEEmembership{Member,~SEIEE~5-321,}}
\maketitle

\begin{abstract}
  
\end{abstract}

% Note that keywords are not normally used for peerreview papers.
\begin{IEEEkeywords}

\end{IEEEkeywords}

\section{Introduction}
\IEEEPARstart{T}{here} is an overlap between image enhancement and image restoration. Although

% \section{Related Work}
% \label{sec:related work}

\section{Wavelet-based Algorithms}
\label{sec:wavelet-based-algorithms}

\subsection{Hard Thresholding in Wavelet Bases}
\label{subsec:hard-thresholding-in-wavelet-bases}

A simple but efficient non-linear denoising estimator is obtained by thresholding the coefficients of $f$
in a well chosen orthogonal basis.
In the following, we will focuss on a wavelet basis, which is efficient to denoise piecewise regular images.
The hard thresholding operator with threshold $T\ge{0}$ applied to some image $f$ is defined as
\begin{eqnarray}
  S_T^0(f)&=&\sum_{\mid{<f,\psi_m>\mid}>T}<f,\psi_m>\psi_m\\
  &=&\sum_{m}s_T^0{(<f,\psi_m>)\psi_m}\nonumber
\end{eqnarray}
where the hard thresholding operator is
\begin{eqnarray}
  s_T^{0}(\alpha)=
  \begin{cases}
    \alpha\ \ \ if\ \mid{\alpha}\mid>T\\
    0\ \ \ otherwise
  \end{cases}
\end{eqnarray}
The denoising estimator is then defined as
\begin{equation}
  \tilde{f}=S_T^0(f)
\end{equation}

\section{Experiment}
\label{sec:experiment}

\subsection{Gaussian Noise}
\label{subsec:gaussian-noise}

For real life applications, we cann't have access to the underlying image $f_0$. We assume that $f_0$
is known, and $f=f_0+w$ is generated using a single realization of the noise $w$.We define the estimated image
as $\tilde{f}$ which is the denoised image. Fig. \ref{fig:add-gaussian-noise} shows images which were added Gaussian noise with standard deviation $0.1$. 

%\usepackage{subcaption}
\begin{figure}[!htb]
  \centering
  \begin{subfigure}[t]{.25\textwidth}
    \centering
    \includegraphics[width=0.86\textwidth]{./code/results/lena512.eps}
    \caption{Lena}
    \label{subfig:lena}
  \end{subfigure}%
  \begin{subfigure}[t]{.25\textwidth}
    \centering
    \includegraphics[width=0.86\textwidth]{./code/results/lena512_gaussian_0_1.eps}
    \caption{Gaussian noisy image of (a),\ $\sigma=0.1$}
    \label{subfig:gaussian-noisy-image-of-a}
  \end{subfigure}
  \begin{subfigure}[t]{0.25\textwidth}
    \centering
    \includegraphics[width=0.86\textwidth]{./code/results/flowr.eps}
    \caption{Flower}
    \label{subfig:flower}
  \end{subfigure}%
  \begin{subfigure}[t]{0.25\textwidth}
    \centering
    \includegraphics[width=0.86\textwidth]{./code/results/flowr_gaussian_0_1.eps}
    \caption{Gaussian noisy image of (c),\ $\sigma=0.1$}
    \label{subfig:gaussian-noisy-image-of-c}
  \end{subfigure}
  \caption{Add Gaussian noise}
  \label{fig:add-gaussian-noise}
\end{figure}

\subsection{Denosing with Hard Thresholding}
\label{subsec:denosing-with-hard-thresholding}



\section{Conclusion}

% use section* for acknowledgment
\section*{Acknowledgment}
I would like to thank Professor Zhang Rui for her instruction.

\begin{thebibliography}{1}
\bibitem{dark-channel}
  He K, Sun J, Tang X. Single image haze removal using dark channel prior[J].
  IEEE transactions on pattern analysis and machine intelligence, 2011, 33(12): 2341-2353.
\end{thebibliography}


% \begin{IEEEbiographynophoto}{Zhou Peng}
% received his bachelor degree from UESTC.He is now studying for a PhD in SJTU.
% \end{IEEEbiographynophoto}

\vfill

\end{document}


