
%% bare_jrnl_comsoc.tex
%% V1.4b
%% 2015/08/26
%% by Michael Shell
%% see http://www.michaelshell.org/
%% for current contact information.
%%
%% This is a skeleton file demonstrating the use of IEEEtran.cls
%% (requires IEEEtran.cls version 1.8b or later) with an IEEE
%% Communications Society journal paper.
%%
%% Support sites:
%% http://www.michaelshell.org/tex/ieeetran/
%% http://www.ctan.org/pkg/ieeetran
%% and
%% http://www.ieee.org/

%%*************************************************************************
%% Legal Notice:
%% This code is offered as-is without any warranty either expressed or
%% implied; without even the implied warranty of MERCHANTABILITY or
%% FITNESS FOR A PARTICULAR PURPOSE! 
%% User assumes all risk.
%% In no event shall the IEEE or any contributor to this code be liable for
%% any damages or losses, including, but not limited to, incidental,
%% consequential, or any other damages, resulting from the use or misuse
%% of any information contained here.
%%
%% All comments are the opinions of their respective authors and are not
%% necessarily endorsed by the IEEE.
%%
%% This work is distributed under the LaTeX Project Public License (LPPL)
%% ( http://www.latex-project.org/ ) version 1.3, and may be freely used,
%% distributed and modified. A copy of the LPPL, version 1.3, is included
%% in the base LaTeX documentation of all distributions of LaTeX released
%% 2003/12/01 or later.
%% Retain all contribution notices and credits.
%% ** Modified files should be clearly indicated as such, including  **
%% ** renaming them and changing author support contact information. **
%%*************************************************************************


% *** Authors should verify (and, if needed, correct) their LaTeX system  ***
% *** with the testflow diagnostic prior to trusting their LaTeX platform ***
% *** with production work. The IEEE's font choices and paper sizes can   ***
% *** trigger bugs that do not appear when using other class files.       ***                          ***
% The testflow support page is at:
% http://www.michaelshell.org/tex/testflow/



\documentclass[journal,comsoc]{IEEEtran}
%
% If IEEEtran.cls has not been installed into the LaTeX system files,
% manually specify the path to it like:
% \documentclass[journal,comsoc]{../sty/IEEEtran}


\usepackage[T1]{fontenc}% optional T1 font encoding
\usepackage{url}
\usepackage{graphicx}
\usepackage{caption}


% Some very useful LaTeX packages include:
% (uncomment the ones you want to load)


% *** MISC UTILITY PACKAGES ***
%
%\usepackage{ifpdf}
% Heiko Oberdiek's ifpdf.sty is very useful if you need conditional
% compilation based on whether the output is pdf or dvi.
% usage:
% \ifpdf
%   % pdf code
% \else
%   % dvi code
% \fi
% The latest version of ifpdf.sty can be obtained from:
% http://www.ctan.org/pkg/ifpdf
% Also, note that IEEEtran.cls V1.7 and later provides a builtin
% \ifCLASSINFOpdf conditional that works the same way.
% When switching from latex to pdflatex and vice-versa, the compiler may
% have to be run twice to clear warning/error messages.






% *** CITATION PACKAGES ***
%
%\usepackage{cite}
% cite.sty was written by Donald Arseneau
% V1.6 and later of IEEEtran pre-defines the format of the cite.sty package
% \cite{} output to follow that of the IEEE. Loading the cite package will
% result in citation numbers being automatically sorted and properly
% "compressed/ranged". e.g., [1], [9], [2], [7], [5], [6] without using
% cite.sty will become [1], [2], [5]--[7], [9] using cite.sty. cite.sty's
% \cite will automatically add leading space, if needed. Use cite.sty's
% noadjust option (cite.sty V3.8 and later) if you want to turn this off
% such as if a citation ever needs to be enclosed in parenthesis.
% cite.sty is already installed on most LaTeX systems. Be sure and use
% version 5.0 (2009-03-20) and later if using hyperref.sty.
% The latest version can be obtained at:
% http://www.ctan.org/pkg/cite
% The documentation is contained in the cite.sty file itself.






% *** GRAPHICS RELATED PACKAGES ***
%
\ifCLASSINFOpdf
  % \usepackage[pdftex]{graphicx}
  % declare the path(s) where your graphic files are
  % \graphicspath{{../pdf/}{../jpeg/}}
  % and their extensions so you won't have to specify these with
  % every instance of \includegraphics
  % \DeclareGraphicsExtensions{.pdf,.jpeg,.png}
\else
  % or other class option (dvipsone, dvipdf, if not using dvips). graphicx
  % will default to the driver specified in the system graphics.cfg if no
  % driver is specified.
  % \usepackage[dvips]{graphicx}
  % declare the path(s) where your graphic files are
  % \graphicspath{{../eps/}}
  % and their extensions so you won't have to specify these with
  % every instance of \includegraphics
  % \DeclareGraphicsExtensions{.eps}
\fi
% graphicx was written by David Carlisle and Sebastian Rahtz. It is
% required if you want graphics, photos, etc. graphicx.sty is already
% installed on most LaTeX systems. The latest version and documentation
% can be obtained at: 
% http://www.ctan.org/pkg/graphicx
% Another good source of documentation is "Using Imported Graphics in
% LaTeX2e" by Keith Reckdahl which can be found at:
% http://www.ctan.org/pkg/epslatex
%
% latex, and pdflatex in dvi mode, support graphics in encapsulated
% postscript (.eps) format. pdflatex in pdf mode supports graphics
% in .pdf, .jpeg, .png and .mps (metapost) formats. Users should ensure
% that all non-photo figures use a vector format (.eps, .pdf, .mps) and
% not a bitmapped formats (.jpeg, .png). The IEEE frowns on bitmapped formats
% which can result in "jaggedy"/blurry rendering of lines and letters as
% well as large increases in file sizes.
%
% You can find documentation about the pdfTeX application at:
% http://www.tug.org/applications/pdftex





% *** MATH PACKAGES ***
%
\usepackage{amsmath}
% A popular package from the American Mathematical Society that provides
% many useful and powerful commands for dealing with mathematics.
% Do NOT use the amsbsy package under comsoc mode as that feature is
% already built into the Times Math font (newtxmath, mathtime, etc.).
% 
% Also, note that the amsmath package sets \interdisplaylinepenalty to 10000
% thus preventing page breaks from occurring within multiline equations. Use:
\interdisplaylinepenalty=2500
% after loading amsmath to restore such page breaks as IEEEtran.cls normally
% does. amsmath.sty is already installed on most LaTeX systems. The latest
% version and documentation can be obtained at:
% http://www.ctan.org/pkg/amsmath


% Select a Times math font under comsoc mode or else one will automatically
% be selected for you at the document start. This is required as Communications
% Society journals use a Times, not Computer Modern, math font.
\usepackage[cmintegrals]{newtxmath}
% The freely available newtxmath package was written by Michael Sharpe and
% provides a feature rich Times math font. The cmintegrals option, which is
% the default under IEEEtran, is needed to get the correct style integral
% symbols used in Communications Society journals. Version 1.451, July 28,
% 2015 or later is recommended. Also, do *not* load the newtxtext.sty package
% as doing so would alter the main text font.
% http://www.ctan.org/pkg/newtx
%
% Alternatively, you can use the MathTime commercial fonts if you have them
% installed on your system:
%\usepackage{mtpro2}
%\usepackage{mt11p}
%\usepackage{mathtime}


%\usepackage{bm}
% The bm.sty package was written by David Carlisle and Frank Mittelbach.
% This package provides a \bm{} to produce bold math symbols.
% http://www.ctan.org/pkg/bm





% *** SPECIALIZED LIST PACKAGES ***
%
%\usepackage{algorithmic}
% algorithmic.sty was written by Peter Williams and Rogerio Brito.
% This package provides an algorithmic environment fo describing algorithms.
% You can use the algorithmic environment in-text or within a figure
% environment to provide for a floating algorithm. Do NOT use the algorithm
% floating environment provided by algorithm.sty (by the same authors) or
% algorithm2e.sty (by Christophe Fiorio) as the IEEE does not use dedicated
% algorithm float types and packages that provide these will not provide
% correct IEEE style captions. The latest version and documentation of
% algorithmic.sty can be obtained at:
% http://www.ctan.org/pkg/algorithms
% Also of interest may be the (relatively newer and more customizable)
% algorithmicx.sty package by Szasz Janos:
% http://www.ctan.org/pkg/algorithmicx




% *** ALIGNMENT PACKAGES ***
%
%\usepackage{array}
% Frank Mittelbach's and David Carlisle's array.sty patches and improves
% the standard LaTeX2e array and tabular environments to provide better
% appearance and additional user controls. As the default LaTeX2e table
% generation code is lacking to the point of almost being broken with
% respect to the quality of the end results, all users are strongly
% advised to use an enhanced (at the very least that provided by array.sty)
% set of table tools. array.sty is already installed on most systems. The
% latest version and documentation can be obtained at:
% http://www.ctan.org/pkg/array


% IEEEtran contains the IEEEeqnarray family of commands that can be used to
% generate multiline equations as well as matrices, tables, etc., of high
% quality.




% *** SUBFIGURE PACKAGES ***
%\ifCLASSOPTIONcompsoc
%  \usepackage[caption=false,font=normalsize,labelfont=sf,textfont=sf]{subfig}
%\else
%  \usepackage[caption=false,font=footnotesize]{subfig}
%\fi
% subfig.sty, written by Steven Douglas Cochran, is the modern replacement
% for subfigure.sty, the latter of which is no longer maintained and is
% incompatible with some LaTeX packages including fixltx2e. However,
% subfig.sty requires and automatically loads Axel Sommerfeldt's caption.sty
% which will override IEEEtran.cls' handling of captions and this will result
% in non-IEEE style figure/table captions. To prevent this problem, be sure
% and invoke subfig.sty's "caption=false" package option (available since
% subfig.sty version 1.3, 2005/06/28) as this is will preserve IEEEtran.cls
% handling of captions.
% Note that the Computer Society format requires a larger sans serif font
% than the serif footnote size font used in traditional IEEE formatting
% and thus the need to invoke different subfig.sty package options depending
% on whether compsoc mode has been enabled.
%
% The latest version and documentation of subfig.sty can be obtained at:
% http://www.ctan.org/pkg/subfig




% *** FLOAT PACKAGES ***
%
%\usepackage{fixltx2e}
% fixltx2e, the successor to the earlier fix2col.sty, was written by
% Frank Mittelbach and David Carlisle. This package corrects a few problems
% in the LaTeX2e kernel, the most notable of which is that in current
% LaTeX2e releases, the ordering of single and double column floats is not
% guaranteed to be preserved. Thus, an unpatched LaTeX2e can allow a
% single column figure to be placed prior to an earlier double column
% figure.
% Be aware that LaTeX2e kernels dated 2015 and later have fixltx2e.sty's
% corrections already built into the system in which case a warning will
% be issued if an attempt is made to load fixltx2e.sty as it is no longer
% needed.
% The latest version and documentation can be found at:
% http://www.ctan.org/pkg/fixltx2e


%\usepackage{stfloats}
% stfloats.sty was written by Sigitas Tolusis. This package gives LaTeX2e
% the ability to do double column floats at the bottom of the page as well
% as the top. (e.g., "\begin{figure*}[!b]" is not normally possible in
% LaTeX2e). It also provides a command:
%\fnbelowfloat
% to enable the placement of footnotes below bottom floats (the standard
% LaTeX2e kernel puts them above bottom floats). This is an invasive package
% which rewrites many portions of the LaTeX2e float routines. It may not work
% with other packages that modify the LaTeX2e float routines. The latest
% version and documentation can be obtained at:
% http://www.ctan.org/pkg/stfloats
% Do not use the stfloats baselinefloat ability as the IEEE does not allow
% \baselineskip to stretch. Authors submitting work to the IEEE should note
% that the IEEE rarely uses double column equations and that authors should try
% to avoid such use. Do not be tempted to use the cuted.sty or midfloat.sty
% packages (also by Sigitas Tolusis) as the IEEE does not format its papers in
% such ways.
% Do not attempt to use stfloats with fixltx2e as they are incompatible.
% Instead, use Morten Hogholm'a dblfloatfix which combines the features
% of both fixltx2e and stfloats:
%
% \usepackage{dblfloatfix}
% The latest version can be found at:
% http://www.ctan.org/pkg/dblfloatfix




%\ifCLASSOPTIONcaptionsoff
%  \usepackage[nomarkers]{endfloat}
% \let\MYoriglatexcaption\caption
% \renewcommand{\caption}[2][\relax]{\MYoriglatexcaption[#2]{#2}}
%\fi
% endfloat.sty was written by James Darrell McCauley, Jeff Goldberg and 
% Axel Sommerfeldt. This package may be useful when used in conjunction with 
% IEEEtran.cls'  captionsoff option. Some IEEE journals/societies require that
% submissions have lists of figures/tables at the end of the paper and that
% figures/tables without any captions are placed on a page by themselves at
% the end of the document. If needed, the draftcls IEEEtran class option or
% \CLASSINPUTbaselinestretch interface can be used to increase the line
% spacing as well. Be sure and use the nomarkers option of endfloat to
% prevent endfloat from "marking" where the figures would have been placed
% in the text. The two hack lines of code above are a slight modification of
% that suggested by in the endfloat docs (section 8.4.1) to ensure that
% the full captions always appear in the list of figures/tables - even if
% the user used the short optional argument of \caption[]{}.
% IEEE papers do not typically make use of \caption[]'s optional argument,
% so this should not be an issue. A similar trick can be used to disable
% captions of packages such as subfig.sty that lack options to turn off
% the subcaptions:
% For subfig.sty:
% \let\MYorigsubfloat\subfloat
% \renewcommand{\subfloat}[2][\relax]{\MYorigsubfloat[]{#2}}
% However, the above trick will not work if both optional arguments of
% the \subfloat command are used. Furthermore, there needs to be a
% description of each subfigure *somewhere* and endfloat does not add
% subfigure captions to its list of figures. Thus, the best approach is to
% avoid the use of subfigure captions (many IEEE journals avoid them anyway)
% and instead reference/explain all the subfigures within the main caption.
% The latest version of endfloat.sty and its documentation can obtained at:
% http://www.ctan.org/pkg/endfloat
%
% The IEEEtran \ifCLASSOPTIONcaptionsoff conditional can also be used
% later in the document, say, to conditionally put the References on a 
% page by themselves.




% *** PDF, URL AND HYPERLINK PACKAGES ***
%
%\usepackage{url}
% url.sty was written by Donald Arseneau. It provides better support for
% handling and breaking URLs. url.sty is already installed on most LaTeX
% systems. The latest version and documentation can be obtained at:
% http://www.ctan.org/pkg/url
% Basically, \url{my_url_here}.




% *** Do not adjust lengths that control margins, column widths, etc. ***
% *** Do not use packages that alter fonts (such as pslatex).         ***
% There should be no need to do such things with IEEEtran.cls V1.6 and later.
% (Unless specifically asked to do so by the journal or conference you plan
% to submit to, of course. )


% correct bad hyphenation here
\hyphenation{op-tical net-works semi-conduc-tor}


\begin{document}


%
% paper title
% Titles are generally capitalized except for words such as a, an, and, as,
% at, but, by, for, in, nor, of, on, or, the, to and up, which are usually
% not capitalized unless they are the first or last word of the title.
% Linebreaks \\ can be used within to get better formatting as desired.
% Do not put math or special symbols in the title.
\title{Analysis of DFT and DCT}
%
%
% author names and IEEE memberships
% note positions of commas and nonbreaking spaces ( ~ ) LaTeX will not break
% a structure at a ~ so this keeps an author's name from being broken across
% two lines.
% use \thanks{} to gain access to the first footnote area
% a separate \thanks must be used for each paragraph as LaTeX2e's \thanks
% was not built to handle multiple paragraphs
%

\author{Zhou~Peng,~\IEEEmembership{Member,~SEIEE~5-321,}}

% note the % following the last \IEEEmembership and also \thanks - 
% these prevent an unwanted space from occurring between the last author name
% and the end of the author line. i.e., if you had this:
% 
% \author{....lastname \thanks{...} \thanks{...} }
%                     ^------------^------------^----Do not want these spaces!
%
% a space would be appended to the last name and could cause every name on that
% line to be shifted left slightly. This is one of those "LaTeX things". For
% instance, "\textbf{A} \textbf{B}" will typeset as "A B" not "AB". To get
% "AB" then you have to do: "\textbf{A}\textbf{B}"
% \thanks is no different in this regard, so shield the last } of each \thanks
% that ends a line with a % and do not let a space in before the next \thanks.
% Spaces after \IEEEmembership other than the last one are OK (and needed) as
% you are supposed to have spaces between the names. For what it is worth,
% this is a minor point as most people would not even notice if the said evil
% space somehow managed to creep in.



% The paper headers
% \markboth{Journal of \LaTeX\ Class Files,~Vol.~14, No.~8, August~2015}%
% {Shell \MakeLowercase{\textit{et al.}}: Bare Demo of IEEEtran.cls for IEEE Communications Society Journals}
% The only time the second header will appear is for the odd numbered pages
% after the title page when using the twoside option.
% 
% *** Note that you probably will NOT want to include the author's ***
% *** name in the headers of peer review papers.                   ***
% You can use \ifCLASSOPTIONpeerreview for conditional compilation here if
% you desire.




% If you want to put a publisher's ID mark on the page you can do it like
% this:
%\IEEEpubid{0000--0000/00\$00.00~\copyright~2015 IEEE}
% Remember, if you use this you must call \IEEEpubidadjcol in the second
% column for its text to clear the IEEEpubid mark.



% use for special paper notices
%\IEEEspecialpapernotice{(Invited Paper)}




% make the title area
\maketitle


% As a general rule, do not put math, special symbols or citations
% in the abstract or keywords.
\begin{abstract}
  In this report, we have compared the results of Discrete Fourier Transform (DFT)
  and Discrete Cosine Transform (DCT) of images.The DFT and DCT perform similar functions:
  they both decompose a finite-length discrete-time signal into a sum of scaled basis functions.
  We find that a signal's DCT representation tends to have more of its energy concentrated in
  a small number of coefficients when compared to DFT.This is desirable for an image compression algorithm. 
\end{abstract}

% Note that keywords are not normally used for peerreview papers.
\begin{IEEEkeywords}
DFT,DCT,spectrum,image transform
\end{IEEEkeywords}

\section{Introduction}
% The very first letter is a 2 line initial drop letter followed
% by the rest of the first word in caps.
% 
% form to use if the first word consists of a single letter:
% \IEEEPARstart{A}{demo} file is ....
% 
% form to use if you need the single drop letter followed by
% normal text (unknown if ever used by the IEEE):
% \IEEEPARstart{A}{}demo file is ....
% 
% Some journals put the first two words in caps:
% \IEEEPARstart{T}{his demo} file is ....
% 
% Here we have the typical use of a "T" for an initial drop letter
% and "HIS" in caps to complete the first word.
% \IEEEPARstart{T}{his} demo file is intended to serve as a ``starter file''
% for IEEE Communications Society journal papers produced under \LaTeX\ using
% IEEEtran.cls version 1.8b and later.
% You must have at least 2 lines in the paragraph with the drop letter
% (should never be an issue)
\IEEEPARstart{D}{igital} image processing methods are mainly divided into
two categories: one method is directly processing in the spatial domain,
and the other is transformed to the transform domain to perform image analysis and processing.
In this report,we mainly focus on the transform domain methods.And we compared two transform
methods:DFT and DCT.We choose two images with different gray level of detail and perform
DFT and DCT on them.By choosing different amount of the transform coefficients and then
doing IDFT and IDCT to reconstruct images,we find the DCT can reconstruct an image better
than DFT with only even less coefficients.We think it's because the DFT signal is implicit
periodicity and discontinuities usually occur at the boundaries.As we all know,if there exists
discontinuities in a function,it needs more terms in its DFT or DCT components.However a DCT
always has a continuous extension at the boundaries.Using this property of DCT,we can represent
an image with less coefficients than its original elements.So DCT is usually applicated in
image compressing.

% \hfill mds
% \hfill August 26, 2015

\section{Related Work}
\label{sec:related work}
There are a number of frequency domain transformation methods that can be used to deal with
image processing problems.However, no matter which method is used, our purpose is to make
the image processing task much easier.

\subsection{Continuous Fourier Transform}
\label{subsec:continuous fourier transform}
The Fourier transform pair in the most general form for a continuous signal $x(t)$ is:
\begin{equation}
  \label{cft}
  X(f)=\int_{-\infty}^{\infty}{x(t)e^{-j2\pi{ft}}dt}
\end{equation}
\begin{equation}
  \label{icft}
  x(t)=\int_{-\infty}^{\infty}X(f)e^{j2\pi{ft}}df
\end{equation}
The complex function $X(f)$ can be writen in polar form
\begin{equation}
  \label{x_f_polar}
  X(f)=X_r(f)+jX_i(f)=|X(f)|e^{j\angle{X(f)}}
\end{equation}
in which $|X(f)|=\sqrt{X_r(f)^2+X_i(f)^2}$, $\angle{X(f)}=\arctan{\frac{X_i(f)}{X_r(f)}}$

\subsection{Two-Dimensional Continuous Fourier Transform}
\label{subsec:two-dimensional continuous fourier transform}
Fourier transform can be generalized to higher dimensions. For example, suppose $f(x,y)$
is a two-dimensional continuous function, its Fourier transform can be defined as
\begin{equation}
  \label{two-cft}
  F(u,v)=\int_{-\infty}^{\infty}\int_{-\infty}^{\infty}f(x,y)e^{-j2\pi{(ux+vy)}}dxdy
\end{equation}
And the corresponding inverse transform is
\begin{equation}
  \label{two-icft}
  f(x,y)=\int_{-\infty}^{\infty}\int_{-\infty}^{\infty}F(u,v)e^{j2\pi{(ux+vy)}}dudv
\end{equation}

\section{Principle of DFT and DCT}
\label{sec:principle of dft and dct}

The Discrete Fourier Transform (DFT) converts a finite sequence of equally-spaced samples of
a function into an equivalent-length sequence of equally-spaced samples of the discrete-time
Fourier transform, which is a complex-valued function of
frequency\footnote{\url{https://en.wikipedia.org/wiki/Discrete_Fourier_transform}}.% \usepackage{url}
Because the computer can only handle discrete values, so the continuous Fourier transform can not
be used directly by the computer. The DFT combines digital image processing with computer science
and can be used in many practical applications.

The DCT is often used in signal and image processing, especially for image compression.
Because it has a strong compact spectrum,most of the signal information tends to be concentrated
in a few low-frequency components of the DCT. 

\subsection{DFT}
\label{subsec:sub-dft}
The Discrete Fourier Transform can be treated as the continuous Fourier transform for signals which
is sampled only at N instants with equally space. Let $f(t)$ be the continuous signal. Let N samples
be denoted as $f[0],f[1],\dots,f[k],\dots,f[N-1]$.
The Fourier transform of the original signal $f(t)$ would be
$$F(j\omega)=\int_{-\infty}^{\infty}f(t)e^{-j\omega{t}}dt$$
We could regard each sample $f[k]$ as an impluse having area $f[k]$. Then, since the integrand exists only at
the sample points and the sample interval is $T$:
$$F(j\omega)=\sum_{k=0}^{N-1}f[k]e^{-j\omega{kT}}$$
We could evaluate this for any $\omega$, but with only $N$ samples to start with, only $N$ final outputs will
be significant.\\
Let $\omega=\frac{2\pi}{NT}n$,then we can get


\begin{equation}
  \label{dft-equation}
  F[n]=\sum_{k=0}^{N-1}f[k]e^{-j\frac{2\pi}{N}nk},n=0,\dots,N-1
\end{equation}
$F[n]$ is the Discrete Fourier Transform of the sequence of $f[k]$.\\
We may write this equation in matrix form as Equ(\ref{dft-mat}).

\begin{figure*}[!ht]
\normalsize

\begin{eqnarray}
  \label{dft-mat}
  \begin{bmatrix}
    F[0]\\F[1]\\F[2]\\\vdots\\F[N-1]
  \end{bmatrix}&=&
  \begin{bmatrix}
    1&1&1&\cdots&1\\
    1&W^{1\cdot1}&W^{1\cdot2}&\cdots&W^{1\cdot(N-1)}\\
    1&W^{2\cdot1}&W^{2\cdot2}&\cdots&W^{2\cdot(N-1)}\\
    \vdots&\\
    1&W^{(N-1)\cdot1}&W^{(N-1)\cdot2}&\cdots&W^{(N-1)\cdot(N-1)}
  \end{bmatrix}
  \begin{bmatrix}
    f[0]\\f[1]\\f[2]\\\vdots\\f[N-1]
  \end{bmatrix}\\\nonumber\\
  &&W=e^{-j\frac{2\pi}{N}},W^{N}=1\nonumber
\end{eqnarray}

\hrulefill
\end{figure*}



The inverse Discrete Fourier Transform of $F[n]$ is
\begin{equation}
  \label{idft-equ}
  f[k]=\frac{1}{N}\sum_{n=0}^{N-1}F[n]e^{j\frac{2\pi}{N}nk}
\end{equation}

\subsection{1D-DCT}
\label{subsec:subsec-1d-dct}

\subsubsection{Forward transform}
\label{subsubsec:1d-dct-forward-transform}

\begin{eqnarray}
  \label{equ:1d-dct}
  F(s)=c(s)\sqrt{\frac{2}{M}}\sum_{m=0}^{M-1}f(m)cos[\frac{\pi}{2M}(2m+1)s]\\
  c(s)=
  \begin{cases}
    \frac{1}{\sqrt{2}}&s=0\\
    1&s=1,2,\ldots,M-1
  \end{cases}\nonumber
\end{eqnarray}

\subsubsection{Inverse transform}
\label{subsubsec:1d-idct-transform}

\begin{equation}
  \label{equ:1d-idct}
  f(m)=\sqrt{\frac{2}{M}}\sum_{s=0}^{M-1}c(s)F(s)cos[\frac{\pi}{2M}(2m+1)s]
\end{equation}

\subsection{2D-DCT}
\label{subsec:2d-dct}

\subsubsection{Forward transform}
\label{subsubsec:2d-dct-forward-transform}


\begin{eqnarray}
  \label{equ:2d-dct-forward-transform}
  F(s,t)=c(s)c(t)\frac{2}{\sqrt{MN}}\sum_{m=0}^{M-1}\sum_{n=0}^{N-1}f(m,n)\nonumber\\
  \times\cos{[\frac{\pi(2n+1)t}{2N}]}\cos{[\frac{\pi(2m+1)s}{2M}]}
\end{eqnarray}

\subsubsection{Inverse transform}
\label{subsubsec:2d-idct-forward-transform}

\begin{eqnarray}
  \label{equ:2d-idct-forward-transform}
  f(m,n)=\frac{2}{\sqrt{MN}}\sum_{s=0}^{M-1}\sum_{t=0}^{N-1}c(s)c(t)F(s,t)\nonumber\\
  \times\cos{[\frac{\pi(2n+1)t}{2N}]}\cos{[\frac{\pi(2m+1)s}{2M}]}
\end{eqnarray}

The DCT is similar to the discrete Fourier transform: it transforms a signal or
image from the spatial domain to the frequency domain.DCT could pack the most
information into the fewest coefficients.

\section{Experiment}
\label{sec:label}

\subsection{Fourier Spectrum}
\label{subsec:sub-fourier-spectrum}
We choose two images with different levels of detail. To ensure they are from different level
detail images,we perform DFT for them.Fig.\ref{Fig:spectrum} shows the original images and corresponding
Fourier spectrum.
From Fig.\ref{Fig:spectrum}, we can find that the baboon has much higher frequency components than lena.

\begin{figure}[!ht]
  \centering
  \captionsetup{justification=centering}
  \includegraphics[width=.5\textwidth]{./code/spectrum.pdf}
  \caption{original image and its Fourier spectrum}
\label{Fig:spectrum}
\end{figure}

\subsection{Comparison of DFT and DCT}
\label{subsec:comparison-dft-dct}

Fig.\ref{Fig:comparison-spectrum} shows the DFT and DCT spectrums. It's obviously that
the DCT representation tends to have more of its energy concentrated in a small number
of coefficients when compared to DFT.


\begin{figure}[!ht]
  \centering
  \captionsetup{justification=centering}
  \includegraphics[width=.5\textwidth]{./code/comparison_dft_dct.pdf}
  \caption{Comparison of spectrum,baboon}
\label{Fig:comparison-spectrum}
\end{figure}

\subsection{Mask Matrix}
\label{subsec:mask-matrix}

We retain different amount of the transform coefficients and then use IDFT and IDCT
to get reconstructed images. Because the transform of an image is a matrix, we could
perform an element multiplication for the matrix with a mask matrix in which the
elements are 0 or 1. One reprents using the corresponding coefficient, and vice versa.

Fig.\ref{Fig:dft-mask} shows the DFT mask matrices with different N. Because the
DFT coefficients are conjugate symmetry, so the mask matrices are also symmetry.

\begin{figure}[!ht]
  \centering
  \captionsetup{justification=centering}
  \includegraphics[width=.5\textwidth]{./code/dft_mask.pdf}
  \caption{DFT mask with different N, N = 0 represents using all transform coefficients}
\label{Fig:dft-mask}
\end{figure}

\begin{figure}[!ht]
  \centering
  \captionsetup{justification=centering}
  \includegraphics[width=.5\textwidth]{./code/dct_mask.pdf}
  \caption{DCT mask with different N. N = 0 represents using all transform coefficients}
\label{Fig:dct-mask}
\end{figure}

\subsection{Inverse Transform}
\label{subsec:inverse-transform}
Fig.\ref{Fig:lena0} - Fig.\ref{Fig:baboon512} show the reconstructed images using different
coefficients of DFT and DCT. We can find that the IDCT could reconstruct
images better than IDFT with less number of coefficients. This indicates that the DCT spectrum
is more compacted than DFT spectrum.

\begin{figure}[!ht]
  \centering
  \captionsetup{justification=centering}
  \includegraphics[width=.5\textwidth]{./code/lena0.pdf}
  \caption{Reconstructed using different number of coefficients. N = 0 represents using all
  transform coefficients}
\label{Fig:lena0}
\end{figure}

\begin{figure}[!ht]
  \centering
  \captionsetup{justification=centering}
  \includegraphics[width=.5\textwidth]{./code/lena50.pdf}
  \caption{Reconstructed using different number of coefficients}
\label{Fig:lena50}
\end{figure}

\begin{figure}[!ht]
  \centering
  \captionsetup{justification=centering}
  \includegraphics[width=.5\textwidth]{./code/lena100.pdf}
  \caption{Reconstructed using different number of coefficients}
\label{Fig:lena100}
\end{figure}

\begin{figure}[!ht]
  \centering
  \captionsetup{justification=centering}
  \includegraphics[width=.5\textwidth]{./code/lena200.pdf}
  \caption{Reconstructed using different number of coefficients}
\label{Fig:lena200}
\end{figure}

\begin{figure}[!ht]
  \centering
  \captionsetup{justification=centering}
  \includegraphics[width=.5\textwidth]{./code/lena400.pdf}
  \caption{Reconstructed using different number of coefficients}
\label{Fig:lena400}
\end{figure}

\begin{figure}[!ht]
  \centering
  \captionsetup{justification=centering}
  \includegraphics[width=.5\textwidth]{./code/lena512.pdf}
  \caption{Reconstructed using different number of coefficients}
\label{Fig:lena512}
\end{figure}




\begin{figure}[!ht]
  \centering
  \captionsetup{justification=centering}
  \includegraphics[width=.5\textwidth]{./code/baboon0.pdf}
  \caption{Reconstructed using different number of coefficients. N = 0 represents using all
  transform coefficients}
\label{Fig:baboon0}
\end{figure}

\begin{figure}[!ht]
  \centering
  \captionsetup{justification=centering}
  \includegraphics[width=.5\textwidth]{./code/baboon50.pdf}
  \caption{Reconstructed using different number of coefficients.}
\label{Fig:baboon50}
\end{figure}

\begin{figure}[!ht]
  \centering
  \captionsetup{justification=centering}
  \includegraphics[width=.5\textwidth]{./code/baboon100.pdf}
  \caption{Reconstructed using different number of coefficients.}
\label{Fig:baboon100}
\end{figure}

\begin{figure}[!ht]
  \centering
  \captionsetup{justification=centering}
  \includegraphics[width=.5\textwidth]{./code/baboon200.pdf}
  \caption{Reconstructed using different number of coefficients.}
\label{Fig:baboon200}
\end{figure}

\begin{figure}[!htb]
  \centering
  \captionsetup{justification=centering}
  \includegraphics[width=.5\textwidth]{./code/baboon400.pdf}
  \caption{Reconstructed using different number of coefficients.}
\label{Fig:baboon400}
\end{figure}

\begin{figure}[!htb]
  \centering
  \captionsetup{justification=centering}
  \includegraphics[width=.5\textwidth]{./code/baboon512.pdf}
  \caption{Reconstructed using different number of coefficients.}
\label{Fig:baboon512}
\end{figure}









% An example of a floating figure using the graphicx package.
% Note that \label must occur AFTER (or within) \caption.
% For figures, \caption should occur after the \includegraphics.
% Note that IEEEtran v1.7 and later has special internal code that
% is designed to preserve the operation of \label within \caption
% even when the captionsoff option is in effect. However, because
% of issues like this, it may be the safest practice to put all your
% \label just after \caption rather than within \caption{}.
%
% Reminder: the "draftcls" or "draftclsnofoot", not "draft", class
% option should be used if it is desired that the figures are to be
% displayed while in draft mode.
%
%\begin{figure}[!t]
%\centering
%\includegraphics[width=2.5in]{myfigure}
% where an .eps filename suffix will be assumed under latex, 
% and a .pdf suffix will be assumed for pdflatex; or what has been declared
% via \DeclareGraphicsExtensions.
%\caption{Simulation results for the network.}
%\label{fig_sim}
%\end{figure}

% Note that the IEEE typically puts floats only at the top, even when this
% results in a large percentage of a column being occupied by floats.


% An example of a double column floating figure using two subfigures.
% (The subfig.sty package must be loaded for this to work.)
% The subfigure \label commands are set within each subfloat command,
% and the \label for the overall figure must come after \caption.
% \hfil is used as a separator to get equal spacing.
% Watch out that the combined width of all the subfigures on a 
% line do not exceed the text width or a line break will occur.
%
%\begin{figure*}[!t]
%\centering
%\subfloat[Case I]{\includegraphics[width=2.5in]{box}%
%\label{fig_first_case}}
%\hfil
%\subfloat[Case II]{\includegraphics[width=2.5in]{box}%
%\label{fig_second_case}}
%\caption{Simulation results for the network.}
%\label{fig_sim}
%\end{figure*}
%
% Note that often IEEE papers with subfigures do not employ subfigure
% captions (using the optional argument to \subfloat[]), but instead will
% reference/describe all of them (a), (b), etc., within the main caption.
% Be aware that for subfig.sty to generate the (a), (b), etc., subfigure
% labels, the optional argument to \subfloat must be present. If a
% subcaption is not desired, just leave its contents blank,
% e.g., \subfloat[].


% An example of a floating table. Note that, for IEEE style tables, the
% \caption command should come BEFORE the table and, given that table
% captions serve much like titles, are usually capitalized except for words
% such as a, an, and, as, at, but, by, for, in, nor, of, on, or, the, to
% and up, which are usually not capitalized unless they are the first or
% last word of the caption. Table text will default to \footnotesize as
% the IEEE normally uses this smaller font for tables.
% The \label must come after \caption as always.
%
%\begin{table}[!t]
%% increase table row spacing, adjust to taste
%\renewcommand{\arraystretch}{1.3}
% if using array.sty, it might be a good idea to tweak the value of
% \extrarowheight as needed to properly center the text within the cells
%\caption{An Example of a Table}
%\label{table_example}
%\centering
%% Some packages, such as MDW tools, offer better commands for making tables
%% than the plain LaTeX2e tabular which is used here.
%\begin{tabular}{|c||c|}
%\hline
%One & Two\\
%\hline
%Three & Four\\
%\hline
%\end{tabular}
%\end{table}


% Note that the IEEE does not put floats in the very first column
% - or typically anywhere on the first page for that matter. Also,
% in-text middle ("here") positioning is typically not used, but it
% is allowed and encouraged for Computer Society conferences (but
% not Computer Society journals). Most IEEE journals/conferences use
% top floats exclusively. 
% Note that, LaTeX2e, unlike IEEE journals/conferences, places
% footnotes above bottom floats. This can be corrected via the
% \fnbelowfloat command of the stfloats package.




\section{Conclusion}

The Discrete Fourier Transform (DFT) and Discrete Cosine Transform (DCT) perform similar functions:
they both decompose a finite-length discrete-time vector into a sum of scaled-and-shifted basis functions.
The difference between the two is the type of basis function used by each transform. the DFT uses a set of
complex exponential functions, while the DCT uses only (real-valued) cosine functions.

A signal's DCT representation tends to have more of its energy concentrated in a small number of coefficients
when compared to DFT. This is desirable for a compression algorithm. If you can approximately represent the original
signal using a relatively small set of DCT coefficients, then you can reduce data storage requirement by only storing
the DCT outputs that contain significant amounts of energy.

In particular, it is well known that any discontinuities in a function reduce the rate of convergence of the Fourier series.
The smoother the function is, the fewer terms in its DFT or DCT are required to represent it accurately,
and the more it can be compressed. However, the implicit periodicity of the DFT means that discontinuities usually
occur at the boundaries. In contrast, a DCT always yields a continuous extension at the boundaries.
This is why DCT generally perform better for signal compression than DFT.

% use section* for acknowledgment
\section*{Acknowledgment}
I would like to thank Professor Zhang Rui for her instruction.

\begin{thebibliography}{1}
\bibitem{IEEEhowto:kopka}
Digital Image Processing (Third Edition), Rafael C. Gonzalez, Richard E. Woods, 2011.
\end{thebibliography}

%\begin{IEEEbiography}[{\includegraphics[width=1in,height=1.25in,clip,keepaspectratio]{mshell}}]{Michael Shell}
% or if you just want to reserve a space for a photo:

\begin{IEEEbiography}[{\includegraphics[width=1in,height=1.25in,clip,keepaspectratio]{./picture.jpg}}]{Zhou Peng}
received his bachelor degree from UESTC.He is now studying for a PhD in SJTU.
\end{IEEEbiography}

% if you will not have a photo at all:
% \begin{IEEEbiographynophoto}{John Doe}
% Biography text here.
% \end{IEEEbiographynophoto}

% insert where needed to balance the two columns on the last page with
% biographies
%\newpage

% You can push biographies down or up by placing
% a \vfill before or after them. The appropriate
% use of \vfill depends on what kind of text is
% on the last page and whether or not the columns
% are being equalized.

\vfill


\end{document}


