\documentclass[journal,comsoc]{IEEEtran}
%\usepackage[UTF8, heading = false, scheme = plain]{ctex}
\usepackage[T1]{fontenc}% optional T1 font encoding
\usepackage{url}
\usepackage{graphicx}
\usepackage{caption}
\usepackage{subcaption}
\usepackage{amsmath}
\interdisplaylinepenalty=2500
\usepackage[cmintegrals]{newtxmath}
\hyphenation{op-tical net-works semi-conduc-tor}

\makeatletter
    \setlength\@fptop{0\p@}
\makeatother

\begin{document}
\title{Experiment of Haze Removal Using Dark Channel Prior}
\author{Zhou~Peng,~\IEEEmembership{Member,~SEIEE~5-321,}}
\maketitle

\begin{abstract}
  
  Haze removal is a process that attempts to remove haze from images. In this report,
  I reimplement the process of haze removal by utilizing the dark channel of images.
  The dark channel prior indicates that most local patches in outdoor haze-free images
  contain some pixels whose intensity is very low in some channels. Based on this prior,
  we can remove haze effectively for many kinds of images. In addition, I also attempt
  to modify values of some hyper parameters in the algorithm, and observe the influence
  on the recovered image due to the changes of these parameters.
\end{abstract}

% Note that keywords are not normally used for peerreview papers.
\begin{IEEEkeywords}
Dehaze, image enhancement
\end{IEEEkeywords}

\section{Introduction}
\IEEEPARstart{T}{here} is an overlap between image enhancement and image restoration. Although
we distinguish them by emphasizing objective criteria and subjective standards, the final results
is to improve the quality of images. Image dehaze is a typical representation of these two technologies.
If the haze is regard as a noise, then the standard of removing the haze is clearly very objective.
That is to restore the image to no haze under the situation. However, if taking the haze environment
as an original image, then the dehaze is clearly in order to improve the subjective visual quality
of the image by an enhancement. In this report, I regard the image dehazing operator as a process of
image enhancement. First, I will perform haze removal by using MeituXiuXiu. And then, I will use the
method which is called dark channel prior\cite{dark-channel} to perform haze removal. To get an
insight of the dark channel method, I test many different values of hyper parameters. By comparison
of the recovered images, it's easier to understand the theory of dark channel prior.

% \section{Related Work}
% \label{sec:related work}

\begin{figure}[!hbt]
  \centering
  \begin{subfigure}{0.5\textwidth}
    \centering
    \includegraphics[width=0.9\textwidth]{code/images/wumai.jpg}
    \caption{Original hazy image}
    \label{subfig:hazy-image}
  \end{subfigure}
  \begin{subfigure}{.5\textwidth}
    \centering
    \includegraphics[width=0.9\textwidth]{code/images/wumai_meitu_1.jpg}
    \caption{Haze removal using MeituXiuXiu}
    \label{subfig:haze-removal-using-meituxiuxiu}
  \end{subfigure}
  \begin{subfigure}{.5\textwidth}
    \centering
    \includegraphics[width=0.9\textwidth]{code/results/wumai_dehaze_7.eps}
    \caption{Haze removal using dark channel method}
    \label{subfig:haze-removal-using-dark-channel-methods}
  \end{subfigure}
  \caption{Comparison of dehazing results}
  \label{fig:comparison-of-dehazing-results}
\end{figure}

% %\usepackage{subcaption}
% \begin{figure}[!hbt]
%   \centering
%   \begin{subfigure}{.5\textwidth}
%     \centering
%     \includegraphics[width=0.9\textwidth]{code/images/wumai.jpg}
%     \caption{}
%     \label{subfig:original-image-with-haze}
%   \end{subfigure}\\
%   \begin{subfigure}{.5\textwidth}
%     \centering
%     \includegraphics[width=0.9\textwidth]{code/images/wumai_meitu_1.jpg}
%     \caption{}
%     \label{subfig:haze-removal-by-meituxiuxiu}
%   \end{subfigure}
%   \caption{Haze removal using MeituXiuxiu. (a) Original image with haze.
%     (b) Haze removal by MeituXiuXiu.}
%   \label{fig:haze-removal-using meituxiuxiu}
% \end{figure}




\section{Dark Channel Prior}
\label{sec:dark-channel-prior}
In most of the non-sky local areas, some pixels always have at least one color channel with a very low value.
In other words, the minimum value of the intensity in a region is a small number.

Suppose the input image is $J$.The dark channel of $J$ is given by
\begin{equation}
  \label{eq:dark-channel-prior}
  J^{dark}(x)=\min_{y\in{\Omega(x)}}(\min_{c\in{\{r,g,b\}}}J^c(y))
\end{equation}
where $J^c$ is a color channel of $J$ and $\Omega(x)$ is a local patch centered at $x$.

The dark channel theorem indicates that if $J$ is an outdoor haze-free image,the intensity of $J$'s dark channel
is low and tends to be zero.Fig.\ref{fig:example-of-dark-channel} shows this phenomenon clearly.
\begin{equation}
  \label{eq:dark-channel-tends-to-be-zero}
  J^{dark}\rightarrow0
\end{equation}
%\usepackage{subcaption}
\begin{figure*}[!htb]
  \centering
  \begin{subfigure}{0.5\textwidth}
    \centering
    \includegraphics[width=0.8\textwidth]{code/images/haze_free.jpg}
    \caption{A haze-free image}
    \label{subfig:a-haze-free-image}
  \end{subfigure}%
  \begin{subfigure}{0.5\textwidth}
    \centering
    \includegraphics[width=0.8\textwidth]{code/results/haze_free_dc.eps}
    \caption{Dark channel of (a)}
    \label{subfig:dark-channel-of-c}
  \end{subfigure}

  \begin{subfigure}{.5\textwidth}
    \centering
    \includegraphics[width=0.8\textwidth]{code/images/wumai.jpg}
    \caption{A hazy image}
    \label{subfig:a-hazy-image}
  \end{subfigure}%
  \begin{subfigure}{.5\textwidth}
    \centering
    \includegraphics[width=0.8\textwidth]{code/results/wumai_dc.eps}
    \caption{Dark channel of (c)}
    \label{subfig:dark-channel-of-a}
  \end{subfigure}

  \caption{Examples of dark channel of hazy-free and haze images}
  \label{fig:example-of-dark-channel}
\end{figure*}

\section{Dehaze Algorithm}
\label{sec:dehaze-algorithm}

In computer vision and computer graphics, the following equation described a hazy image is widely used:
\begin{equation}
  \label{eq:hazy-image-model}
  I(x)=J(x)t(x)+A(1-t(x))
\end{equation}
Where $I$ is a image with haze,$J$ is no fog image we want to restore, $A$ is the global atmospheric light component,and $t$ is
the transmission. Now we know $I$ and want to get the target value $J$.Obviously,there are countless solutions of this equations.
Therefore, we need a prior knowledge.

First normalize the equation (\ref{eq:hazy-image-model}):
\begin{equation}
  \label{eq:normalize-hazy-image-model}
  \frac{I^c(x)}{A^c}=t(x)\frac{J^c(x)}{A^c}+1-t(x)
\end{equation}
The $c$ represents each color channel of an image.

Assume that within each window transmission $t$ is a constant denoted by $\tilde{t}(x)$.$A$ has been given.
Then,we calculate the dark channel on both sides of (\ref{eq:normalize-hazy-image-model}).
\begin{equation}
  \label{eq:dark-channel-normalize-hazy-image-model}
  \min_{y\in{\Omega{(x)}}}(\min_c{\frac{I^c(y)}{A^c}})=\tilde{t}(x)\min_{y\in{\Omega{(x)}}}(\min_c{\frac{J^c(y)}{A^c}})+1-\tilde{t}(x)
\end{equation}
In equation (\ref{eq:dark-channel-normalize-hazy-image-model}),$J$ is fog-free image which we want to get.
According to equation (\ref{eq:dark-channel-tends-to-be-zero}),we can get:
\begin{equation}
  \label{eq:zero-dark-channel-normalize-hazy-image-model}
  J^{dark}(x)=\min_{y\in{\Omega{(x)}}}(\min_{c}J^c(y))=0
\end{equation}
Therefore, it can be deduced:
\begin{equation}
  \label{eq:deduced-zero-dark-channel-normalize}
  \min_{y\in{\Omega{(x)}}}(\min_{c}\frac{J^c(y)}{A^c})=0
\end{equation}
Putting (\ref{eq:deduced-zero-dark-channel-normalize}) into (\ref{eq:dark-channel-normalize-hazy-image-model}),
we can estimate the transmission $\tilde{t}(x)$ by
\begin{equation}
  \label{eq:transmission-estimation}
  \tilde{t}(x)=1-\min_{y\in{\Omega{(x)}}}(\min_c{\frac{I^c(y)}{A^c}})
\end{equation}

In addition, the presence of fog makes humans feel the presence of depth of field.Therefore, it is necessary to
retain a certain degree of fog, which can be implemented by introducing a factor between $[0,1]$ in the equation
(\ref{eq:transmission-estimation}).
\begin{equation}
  \label{eq:parameter-transmission-estimation}
  \tilde{t}(x)=1-\omega\min_{y\in{\Omega{(x)}}}(\min_c{\frac{I^c(y)}{A^c}})
\end{equation}

The atmospheric light is calculated using the dark channel and the original image. To ensure that we select the
brightest pixels that are not objects within the image, you must look at the pixels in the dark channel.
The simple approach used in the paper \cite{dark-channel} finds the $0.1\%$ brightest pixels of the dark channel,
then selects the maximum value from those pixels in the original image.

When the $t$'s value is small, $J(x)t(x)$ is very close to zero and the recovered $J$ tends to be noise.So we restrict
the $t(x)$ by a lower bound $t_0$.The final $J(x)$ can be recovered by
\begin{equation}
  \label{eq:final-recovered-jx}
  J(x)=\frac{I(x)-A}{\max{(t(x),t_0)}}+A
\end{equation}
$t_0$ usually is $0.1$.

\section{Experiment}
\label{sec:experiment}

\subsection{Different Window Size}
\label{subsec:different-window-size}

I test many dark channel with different value of window size. Fig.\ref{fig:dark-channel-with-different-window-size} shows the results of
dark chanenl with different window size.Apparently,the greater the size of a window, the greater the dark channel it
contains.Dark channel will be more black.
%\usepackage{subcaption}
\begin{figure}[!htb]
  \centering
  \begin{subfigure}{.5\textwidth}
    \centering
    \includegraphics[width=0.9\textwidth]{code/results/wumai_dc_1.eps}
    \caption{$3\times3$}
    \label{subfig:3x3}
  \end{subfigure}
  \begin{subfigure}{.5\textwidth}
    \centering
    \includegraphics[width=0.9\textwidth]{code/results/wumai_dc_7.eps}
    \caption{$15\times15$}
    \label{subfig:15x15}
  \end{subfigure}
  \begin{subfigure}{0.5\textwidth}
    \centering
    \includegraphics[width=0.9\textwidth]{code/results/wumai_dc_10.eps}
    \caption{$21\times21$}
    \label{subfig:21x21}
  \end{subfigure}
  \begin{subfigure}{0.5\textwidth}
    \centering
    \includegraphics[width=0.9\textwidth]{code/results/wumai_dc_50.eps}
    \caption{$101\times101$}
    \label{subfig:101x101}
  \end{subfigure}
  \caption{Dark channel with different window size}
  \label{fig:dark-channel-with-different-window-size}
\end{figure}

\subsection{Recovered Image}
\label{subsec:recovered-image}

I fix the window size to $15\times15$. Using dark channel dehazing algorithm, I get a
haze-free image like Fig.\ref{fig:haze-removal-results}. I found that the haze removal
method can remove the haze in the crowd effectively. But for the sky, the algorithm is not
so good. As a solution, I added a parameter: the largest global atmospheric value, when the
calculated value is greater than this value, it takes the value. Fig.\ref{fig:limit-of-largest-atmospheric-light-value}
shows the haze-free results by using different largest global atmospheric values. Apparently,
when we set the value to $0.7$, the hog-free image looks better.
\begin{figure}[!htb]
  \centering
  \includegraphics[width=0.5\textwidth]{code/results/wumai_dehaze_10.eps}
  \caption{Haze removal results}
  \label{fig:haze-removal-results}    
\end{figure}


%\usepackage{subcaption}
\begin{figure*}[t]
  \centering
  \begin{subfigure}{.5\textwidth}
    \centering
    \includegraphics[width=0.9\textwidth]{code/results/wumai_dehaze_3.eps}
    \caption{0.3}
    \label{subfig:atmospheric-light-0.3}
  \end{subfigure}%
  \begin{subfigure}{.5\textwidth}
    \centering
    \includegraphics[width=0.9\textwidth]{code/results/wumai_dehaze_5.eps}
    \caption{0.5}
    \label{subfig:atmospheric-light-0.5}
  \end{subfigure}\\
  \begin{subfigure}{0.5\textwidth}
    \centering
    \includegraphics[width=0.9\textwidth]{code/results/wumai_dehaze_6.eps}
    \caption{0.6}
    \label{subfig:atmospheric-light-0.6}
  \end{subfigure}%
  \begin{subfigure}{0.5\textwidth}
    \centering
    \includegraphics[width=0.9\textwidth]{code/results/wumai_dehaze_7.eps}
    \caption{0.7}
    \label{subfig:atmospheric-light-0.7}
  \end{subfigure}\\

  \begin{subfigure}{0.5\textwidth}
    \centering
    \includegraphics[width=0.9\textwidth]{code/results/wumai_dehaze_8.eps}
    \caption{0.8}
    \label{subfig:atmospheric-light-0.8}
  \end{subfigure}
  \caption{Limit of largest atmospheric light value}
  \label{fig:limit-of-largest-atmospheric-light-value}
\end{figure*}

\subsection{Comparison With MeituXiuXiu}
\label{subsec:comparison-with-meituxiuxiu}

Fig.\ref{fig:comparison-of-dehazing-results} shows the original hazy image and the
corresponding dehazing images by using different methods. I find the result of
MeituXiuXiu can only remove some haze in the crowd. The dark channel method can
make the entire image look better.
%\usepackage{subcaption}

\section{Conclusion}

Dark channel prior is a statistical result for a lot of outdoor photos. If the target scene are
similar to the atmosphere light such as snow, white wall and sea, due to the premise condition is not
correct, so it's generally unable to obtain satisfactory results. But for general scenery images,
this algorithm can work well.

% use section* for acknowledgment
\section*{Acknowledgment}
I would like to thank Professor Zhang Rui for her instruction.

\begin{thebibliography}{1}
\bibitem{dark-channel}
  He K, Sun J, Tang X. Single image haze removal using dark channel prior[J].
  IEEE transactions on pattern analysis and machine intelligence, 2011, 33(12): 2341-2353.
\end{thebibliography}


% \begin{IEEEbiographynophoto}{Zhou Peng}
% received his bachelor degree from UESTC.He is now studying for a PhD in SJTU.
% \end{IEEEbiographynophoto}

\vfill

\end{document}


