\documentclass[journal,comsoc]{IEEEtran}

\usepackage[T1]{fontenc}% optional T1 font encoding
\usepackage{url}
\usepackage{graphicx}
\usepackage{caption}
\usepackage{amsmath}
\interdisplaylinepenalty=2500
\usepackage[cmintegrals]{newtxmath}
\hyphenation{op-tical net-works semi-conduc-tor}

\begin{document}
\title{Analysis of DFT and DCT}
\author{Zhou~Peng,~\IEEEmembership{Member,~SEIEE~5-321,}}
\maketitle

\begin{abstract}
  In this report, we have compared the results of Discrete Fourier Transform (DFT)
  and Discrete Cosine Transform (DCT) of images.The DFT and DCT perform similar functions:
  they both decompose a finite-length discrete-time signal into a sum of scaled basis functions.
  We find that a signal's DCT representation tends to have more of its energy concentrated in
  a small number of coefficients when compared to DFT.This is desirable for an image compression algorithm. 
\end{abstract}

% Note that keywords are not normally used for peerreview papers.
\begin{IEEEkeywords}
DFT,DCT,spectrum,image transform
\end{IEEEkeywords}

\section{Introduction}
\IEEEPARstart{D}{igital} image processing methods are mainly divided into
two categories: one method is directly processing in the spatial domain,
and the other is transformed to the transform domain to perform image analysis and processing.
In this report,we mainly focus on the transform domain methods.And we compared two transform
methods:DFT and DCT.We choose two images with different gray level of detail and perform
DFT and DCT on them.By choosing different amount of the transform coefficients and then
doing IDFT and IDCT to reconstruct images,we find the DCT can reconstruct an image better
than DFT with only even less coefficients.We think it's because the DFT signal is implicit
periodicity and discontinuities usually occur at the boundaries.As we all know,if there exists
discontinuities in a function,it needs more terms in its DFT or DCT components.However a DCT
always has a continuous extension at the boundaries.Using this property of DCT,we can represent
an image with less coefficients than its original elements.So DCT is usually applicated in
image compressing.

\section{Related Work}
\label{sec:related work}
There are a number of frequency domain transformation methods that can be used to deal with
image processing problems.However, no matter which method is used, our purpose is to make
the image processing task much easier.

\subsection{Continuous Fourier Transform}
\label{subsec:continuous fourier transform}
The Fourier transform pair in the most general form for a continuous signal $x(t)$ is:
\begin{equation}
  \label{cft}
  X(f)=\int_{-\infty}^{\infty}{x(t)e^{-j2\pi{ft}}dt}
\end{equation}
\begin{equation}
  \label{icft}
  x(t)=\int_{-\infty}^{\infty}X(f)e^{j2\pi{ft}}df
\end{equation}
The complex function $X(f)$ can be writen in polar form
\begin{equation}
  \label{x_f_polar}
  X(f)=X_r(f)+jX_i(f)=|X(f)|e^{j\angle{X(f)}}
\end{equation}
in which $|X(f)|=\sqrt{X_r(f)^2+X_i(f)^2}$, $\angle{X(f)}=\arctan{\frac{X_i(f)}{X_r(f)}}$


\section{Conclusion}

The Discrete Fourier Transform (DFT) and Discrete Cosine Transform (DCT) perform similar functions:
they both decompose a finite-length discrete-time vector into a sum of scaled-and-shifted basis functions.
The difference between the two is the type of basis function used by each transform. the DFT uses a set of
complex exponential functions, while the DCT uses only (real-valued) cosine functions.

% use section* for acknowledgment
\section*{Acknowledgment}
I would like to thank Professor Zhang Rui for her instruction.

\begin{thebibliography}{1}
\bibitem{IEEEhowto:kopka}
Digital Image Processing (Third Edition), Rafael C. Gonzalez, Richard E. Woods, 2011.
\end{thebibliography}


% \begin{IEEEbiographynophoto}{Zhou Peng}
% received his bachelor degree from UESTC.He is now studying for a PhD in SJTU.
% \end{IEEEbiographynophoto}

\vfill

\end{document}


