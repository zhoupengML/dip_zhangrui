\documentclass[journal,comsoc]{IEEEtran}
% \usepackage[UTF8, heading = false, scheme = plain]{ctex}
% Put IEEEtran.cls in the same directory
\usepackage[T1]{fontenc}% optional T1 font encoding
\usepackage{url}
\usepackage{graphicx}
\usepackage{caption}
\usepackage{subcaption}
\usepackage{amsmath}
\interdisplaylinepenalty=2500
\usepackage[cmintegrals]{newtxmath}
\hyphenation{op-tical net-works semi-conduc-tor}
\makeatletter
\setlength\@fptop{0\p@}
\makeatother

\begin{document}
\title{Research on Morphological Filtering Algorithms}
\author{Zhou~Peng,~\IEEEmembership{016034910052,~SEIEE~5-321,}}
\maketitle

\begin{abstract}
  Mathematical morphology provides an approach to the processing of digital images
  that is based on the spatial structure of objects.
  In this report, I explore four different morphological operations: dilation, erosion,
  opening and closing operations to get a better segmentation for a salt-and-pepper
  noisy\cite{noise} image. Results of the experiments show that noise can be effectively removed from
  binary images using combinations of erosion and dilation operations. 
\end{abstract}

% Note that keywords are not normally used for peerreview papers.
\begin{IEEEkeywords}
Morphology,dilation,erosion,opening,closing
\end{IEEEkeywords}

% \begin{IEEEkeywords}
% Morphology,dilation,erosion,opening,closing,segmentation
% \end{IEEEkeywords}

\section{Introduction}

\IEEEPARstart{M}athematical morphological provides an approach to the processing of digital images
that is based on the nature structure of objects in a scene. 
Additionally, edge detection and morphological technique were applied to obtain a
binary mask of the boundary too.

Dilation and erosion operations are fundamental to morphological processing.
Dilation adds pixels to the boundaries of objects in an image, while erosion
removes pixels on object boundaries. In this paper, the combinations of dilate-erode
operations are used to process the image.

%\usepackage{subcaption}
\begin{figure}[!htb]
  \centering
  \begin{subfigure}[t]{.5\textwidth}
    \centering
    \includegraphics[width=0.63\textwidth]{./code/results/coins_original.eps}
    \setcaptionwidth{0.63\textwidth}
    \caption{Original image}
    \label{subfig:./code/results/coins_original.eps}
  \end{subfigure}
  \begin{subfigure}[t]{.5\textwidth}
    \centering
    \includegraphics[width=0.63\textwidth]{./code/results/coins_noise.eps}
    \setcaptionwidth{0.63\textwidth}
    \caption{Add salt-and-pepper noise}
    \label{subfig:./code/results/coins_noise.eps}
  \end{subfigure}
  \begin{subfigure}[t]{.5\textwidth}
    \centering
    \includegraphics[width=0.63\textwidth]{./code/results/coins_direct_segmentation.eps}
    \setcaptionwidth{0.63\textwidth}
    \caption{Segmentation for (b)}
    \label{subfig:./code/results/coins_direct_segmentation.eps}
  \end{subfigure}%

  \caption{Segmentation for noisy image}
  \label{fig:Segmentation for noisy image}
\end{figure}


\subsection{Dilation}

Dilation is used to add pixels at region boundaries or to fill in holes in the image. 
The dilation of $A$ by $B$ denoted $A\oplus{}B$, is defined as
\begin{equation}
A\oplus{B}=\{Z\mid{}(\hat{B})z\cap{A}\ne\emptyset\}
\end{equation}
where $\emptyset$ is the empty set and $B$ is the structuring element.
In other words, dilation $A$ by $B$ is the set consisting of all the
structuring element origin location. Dilation is applied to the image to
link the pixels as shown in the image.

\subsection{Erosion}

Erosion does the opposite operation of dilation. While dilation expands
boundaries and fills holes, erosion reduces boundaries and increases size of holes.
Erosion reduces boundaries and increases size of holes. It will set ON pixel to OFF
if the structuring element does not completely overlap ON valued pixels. The erosion
of $A$ by $B$ denoted $A\ominus{}B$, is defined as
\begin{equation}
A\ominus{}B=\{Z\mid{}(\hat{B})z\subseteq{A}\}
\end{equation}
Erosion can also be used to reduce noises from an image.

\subsection{Opening and Closing}

Erosion and dilation can be performed repeatedly on an image until the
desired output is produced. However, the order of the operations can
make a difference to the processed image. Two common operations that
combine erosion and dilation are known as opening and closing.
Opening operation involves erosion followed by dilation while closing
operation starts with dilation followed by erosion. Opening smoothes
the contours of objects, breaks narrow joints and eliminates thin protrusions.
Closing produces the smoothing of sections of contours but it fuses narrow breaks,
fills gaps in the contour and eliminates small holes. Opening operation is used
when the image has many small noise regions. Closing, on the other hand, is used
to restore connectivity between close proximity objects.


% \section{Related Work}
% \label{sec:related work}

\section{Experiment}
\label{sec:experiment}

\subsection{Segmentation for Noisy Image}

I add salt-and-pepper noise\cite{noise} to the image which the noise density is $0.02$.
And then I use the Otsu method\cite{Otsu} to segment the noisy image.
Fig. \ref{fig:Segmentation for noisy image} shows the results.


\subsection{Dilation}

The generalization of dilation to gray-level images is that the gray-level value at any point
is replaced by the maximum intensity value covered by the flat structuring element.
Fig. \ref{subfig:./code/results/coins_dilate_segmentation.eps} shows the results of dilation.
We can find that the dilation operation could reduce pepper noise and make the salt noise
become bigger.

%\usepackage{subcaption}
\begin{figure}[!htb]
  \centering
  \begin{subfigure}[t]{.5\textwidth}
    \centering
    \includegraphics[width=0.63\textwidth]{./code/results/coins_dilate_segmentation.eps}
    \setcaptionwidth{0.63\textwidth}
  \end{subfigure}
  \caption{Dilation of Fig. \ref{subfig:./code/results/coins_direct_segmentation.eps}}
  \label{subfig:./code/results/coins_dilate_segmentation.eps}
\end{figure}

\subsection{Erosion}

Erosion of gray-level images is defined in a similar manner to dilation: the gray-level
value at any point is replaced by the minimum intensity value covered by the flat structuring element.
Fig. \ref{subfig:./code/results/coins_erode_segmentation.eps} shows that the erosion operation
could reduce the salt noise but make the pepper noise more intensive.

%\usepackage{subcaption}
\begin{figure}[!htb]
  \centering
  \begin{subfigure}[t]{.5\textwidth}
    \centering
    \includegraphics[width=0.63\textwidth]{./code/results/coins_erode_segmentation.eps}
    \setcaptionwidth{0.63\textwidth}
  \end{subfigure}
  \caption{Erosion of Fig. \ref{subfig:./code/results/coins_direct_segmentation.eps}}
  \label{subfig:./code/results/coins_erode_segmentation.eps}
\end{figure}

\subsection{Morphologically open image}

I perform morphological opening on the binary image Fig. \ref{subfig:./code/results/coins_direct_segmentation.eps}.
Fig. \ref{subfig:./code/results/coins_open_segmentation.eps} shows the result. It's clear that
the opening operation could reduce the salt noise effectively.

%\usepackage{subcaption}
\begin{figure}[!htb]
  \centering
  \begin{subfigure}[t]{.5\textwidth}
    \centering
    \includegraphics[width=0.63\textwidth]{./code/results/coins_open_segmentation.eps}
    \setcaptionwidth{0.63\textwidth}
  \end{subfigure}
  \caption{Morphological opening operation on Fig. \ref{subfig:./code/results/coins_direct_segmentation.eps}}
  \label{subfig:./code/results/coins_open_segmentation.eps}
\end{figure}

\subsection{Morphologically close image}

Analogous to opening operation, I perform morphological closing operation on the binary image
and find that it's very useful for the closing operation to reduce the pepper noise.
Fig. \ref{subfig:./code/results/coins_close_segmentation.eps} shows our results.

%\usepackage{subcaption}
\begin{figure}[h]
  \centering
  \begin{subfigure}[t]{.5\textwidth}
    \centering
    \includegraphics[width=0.63\textwidth]{./code/results/coins_close_segmentation.eps}
    \setcaptionwidth{0.63\textwidth}
  \end{subfigure}
  \caption{Morphological closing operation on Fig. \ref{subfig:./code/results/coins_direct_segmentation.eps}}
  \label{subfig:./code/results/coins_close_segmentation.eps}
\end{figure}

\subsection{Combining Opening and Closing Operation}

To reduce the salt-and-pepper noise at the same time, I perform the opening operation followed
by a closing operation. \ref{subfig:./code/results/coins_open_and_close_segmentation.eps} shows
the results. We can find that the salt-and-pepper noise was reduced completely.

%\usepackage{subcaption}
\begin{figure}[!htb]
  \centering
  \begin{subfigure}[t]{.5\textwidth}
    \centering
    \includegraphics[width=0.63\textwidth]{./code/results/coins_open_and_close_segmentation.eps}
    \setcaptionwidth{0.63\textwidth}
  \end{subfigure}
  \caption{Opening followed by a closing operation}
  \label{subfig:./code/results/coins_open_and_close_segmentation.eps}
\end{figure}


\section{Conclusion}

This report experiments four different morphological operations: dilation, erosion, opening, closing operations.
Morphological operations in this report are used for noise removal and to produce the best results of segmentation.
Dilation is used to enlarge boundaries and closing holes, while erosion does the opposite. Opening performs erosion
followed by dilation and closing, vice versa. Each operation may be applied repeatedly on the image to produce the
best desired result. For example, an image may be subjected to dilation, erosion, erosion and dilation to produce the intended result.

\section*{Acknowledgment}
I would like to thank Professor Zhang Rui for her instruction.

\begin{thebibliography}{1}
\bibitem{Otsu}
  Otsu, Nobuyuki. "A threshold selection method from gray-level histograms." Automatica 11.285-296 (1975): 23-27.
\bibitem{noise}
  S. Umbaugh, Computer Vision and Image Processing. Prentice Hall Inc. 1999.
\end{thebibliography}

% \begin{IEEEbiographynophoto}{Zhou Peng}
%   received his bachelor degree from UESTC.He is now studying for a PhD in SJTU.
% \end{IEEEbiographynophoto}

\vfill

\end{document}



